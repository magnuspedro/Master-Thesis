%%%% GLOSSÁRIO (ELEMENTO OPCIONAL)
%%
%% Relação de palavras ou expressões técnicas de uso restrito, ou de sentido
%% obscuro, utilizadas no texto, acompanhadas das respectivas definições.

%% Glossário (inserir itens em ordem alfabética)
\begin{Glossary}%[\bfseries]%% Estilo de fonte do termo
\item[biber] substituto do Bib\TeX\ para usuários do Bib\LaTeX.
  \hyperpage{40}.
\item[Bib\LaTeX] reimplementação completa das facilidades bibliográficas fornecidas pelo \LaTeX.
  \hyperpage{27}, \hyperpage{40}, \hyperpage{51}.
\item[Bib\LaTeX-abnt] pacote que oferece um estilo Bib\LaTeX\ que atende às regras da ABNT\@.
  \hyperpage{27}, \hyperpage{40}.
\item[Bib\TeX] aplicativo de gerenciamento de referências para a formatação de listas de referências no \LaTeX.
  \hyperpage{40}, \hyperpage{51}.
\glossaryspace%
\item[componente] outro exemplo de uma entrada secundária (componente), subentrada da primária chamada pai; trata-se de uma entrada irmã de outra também secundária chamada filho.
  \hyperpage{41}, \hyperpage{51}.
\glossaryspace%
\item[dissertação] trabalho acadêmico desenvolvido no mestrado.
  \hyperpage{20}, \hyperpage{28}.
\glossaryspace%
\item[equilíbrio da configuração] consistência entre os componentes.
  \hyperpage{41}.
\glossaryspace%
\item[filho] exemplo de uma entrada secundária (filho), subentrada da primária chamada pai.
  \hyperpage{41}, \hyperpage{51}.
\glossaryspace%
\item[\LaTeX] conjunto de macros para o processador de textos \TeX, utilizado amplamente para a produção de textos matemáticos e científicos devido à sua alta qualidade tipográfica.
  \hyperpage{20, 21}, \hyperpage{23}, \hyperpage{29--31}, \hyperpage{37}, \hyperpage{40, 41}, \hyperpage{44--46}, \hyperpage{51}, \hyperpage{54}.
\glossaryspace%
\item[memoir] classe \LaTeX\ que permite a composição de poesia, ficção, obras de não ficção e matemáticas, como livros, relatórios, artigos ou manuscritos.
  \hyperpage{20}, \hyperpage{25}, \hyperpage{32}, \hyperpage{45}.
\glossaryspace%
\item[pai] exemplo de entrada primária (pai) que possui subentradas ou entradas secundárias (filhos).
  \hyperpage{41}, \hyperpage{51}.
\glossaryspace%
\item[tese] trabalho acadêmico desenvolvido no doutorado.
  \hyperpage{20}, \hyperpage{28}.
\item[\TeX] sistema de tipografia criado por Donald E. Knuth.
  \hyperpage{21}, \hyperpage{41}, \hyperpage{45}, \hyperpage{51}, \hyperpage{54}.
\glossaryspace%
\item[\UTFPR-Thesis] modelo \LaTeX\ que permite atender os requisitos das normas definidas pela UTFPR para elaboração de trabalhos acadêmicos.
  \hyperpage{20}, \hyperpage{25}, \hyperpage{27}, \hyperpage{41}.
\end{Glossary}
