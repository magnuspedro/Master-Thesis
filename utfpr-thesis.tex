%%%% utfpr-thesis.tex, 2024/02/10
%%%% Copyright (C) 2021-2024 Luiz E. M. Lima (luizeduardomlima@gmail.com)
%%
%% This work may be distributed and/or modified under the conditions of the
%% LaTeX Project Public License, either version 1.3 of this license or (at your
%% option) any later version.
%% The latest version of this license is in
%%   http://www.latex-project.org/lppl.txt
%% and version 1.3 or later is part of all distributions of LaTeX version
%% 2005/12/01 or later.
%%
%% This work has the LPPL maintenance status `maintained'.
%%
%% The Current Maintainer of this work is Luiz E. M. Lima.
%%
%% This work consists of all files listed in readme.html.
%%
%% A brief description of this work is in readme.txt.

%% Detecção e aviso sobre comandos obsoletos
% \RequirePackage[l2tabu, orthodox]{nag}

%% Classe de documento
\documentclass[%% Opções (^ = padrão; > = para pacotes; ¹ = somente twoside):
  a4paper,%% Tamanho de papel: a4paper, letterpaper (^), etc.
  12pt,%% Tamanho de fonte: 10pt (^), 11pt, 12pt, etc.
  oneside,%% Impressão de folhas: oneside ou twoside (^)
%   openany,%% Impressão de capítulos (¹): openany ou openright (^)
%   fleqn,%% Alinhamento de equações à esquerda (centralizado por padrão)
%   draft,%% Aparência de documento (>): draft ou final (^)
  brazilian,%% Idioma secundário (penúltimo) (>)
  english,%% Idioma primário (último) (>)
]{memoir}

%% Pacotes utilizados
\usepackage[%% Opções (^ = padrão; ¹ = somente twoside; ² = somente openany):
%   Font    = Times,%% Fonte principal: Arial (^), CM (padrão TeX) ou Times
%   Link    = TextColor,%% Cor de hyperlinks: DarkBlue (^) ou TextColor
%   Caption = Left,%% Alinhamento de legendas: Center (^) ou Left
%   Source  = Left,%% Alinhamento de fonte (referência): Center (^) ou Left
%   ABNTCit = NSB,%% Citação ABNT: AAY (Nome, Ano) (^), NRB (1) ou NSB [1]
%   BibDOI  = Icon,%% Ícone de DOI em referências: Icon ou Name (^)
%   BibURL  = Icon,%% Ícone de URL em referências: Icon ou URL (^)
%   CoverBG = On,%% Plano de fundo (capa e contracapa): On ou Off (^)
%   CoverId = All,%% Dados de identificação (capa): All ou Main (^)
%   PageNum = All,%% Numeração de páginas (partes): All, Main (^) ou None
%   TwoSide = All,%% Frente e verso (partes) (¹): All, Main (^) ou None
%   OpenPg  = Odd,%% Paginação de elementos (²): Odd ou Any (^)
%   Version = Defense,%% Versão de documento: Final (^) ou Defense
]{utfpr-thesis}

%% Informações do documento: descomentar para alterar
%%%% Título: {Em Português}; {In English}
\Title{%
    RMT 2.0: Uma ferramenta para identificar e aplicar padrões de projetos fundamentada em microsserviços
}{%
   RMT 2.0: A Tool for the identification and application of Design Patterns Based in Microservices%
}
%%%% Subtítulo (opcional): {Em Português}; {In English}
% \SubTitle{%
%   subtítulo de trabalho acadêmico%
% }{%
%   subtitle of academic work%
% }
%%%% Aluno(s) (de 1 a 5): {Número}; {Dados}
\Student{1}{%
   Gender   = {Male},%% Ou {Female}
   Forename = {Pedro Magnus Pedroso},%% Exceto último e sufixo (e.g., {José Santos da})
   Surname  = {Nogueira},%% Último e sufixo (e.g., {Silva Júnior})
%   Email    = {student1@domain},%% Opcional
%   Lattes   = {0000000000010001},%% Opcional
%   ORCID    = {0000-0000-0001-0001},%% Opcional (CHKTEX 8)
}
%%%% Grau acadêmico (opção): [Número] (Obs.: (¹) automático para cada opção)
% \AcademicDegreeOption[1]%% Doutorado
 \AcademicDegreeOption[2]%% Mestrado
% \AcademicDegreeOption[3]%% Especialização
% \AcademicDegreeOption[4]%% Bacharelado
% \AcademicDegreeOption[5]%% Licenciatura
% \AcademicDegreeOption[6]%% Tecnologia
%%%%%% Grau acadêmico (¹): {Em Português}; {In English}
% \AcademicDegree{Doutorado}{Doctorate}
%%%%%% Título acadêmico (¹): {Em Português}; {In English}
% \AcademicTitle{Doutor\Gen{a}}{Doctor}
%%%%%% Tipo de documento (¹): {Em Português}; {In English}
\DocumentType{Tese}{Thesis}
%%%% Área de concentração (Doutorado e Mestrado): {Em Português}; {In English}
\ConcentrationArea{Sistemas de informação e computação}{Information and Computer Systems}
%%%% Curso(s): {Em Português}; {In English}
\Course{Ciência da Computação}{Computer Science}
%%%% Departamento(s), coordenação(ões) ou programa: {Em Português}; {In English}
\Department{%
   Programa de Pós-Graduação em Ciência da Computação%
}{%
   Computer Science Graduate Program%
}
%%%% Campus: {Cidade}
\Campus{Ponta Grossa}
%%%% Ano(s) (atual por padrão): [De Depósito] (opcional); {De Defesa}
\Year[2024]{2024}

%% Processamento de entradas (itens) de listas, glossários e índices (makeindex)
%% Comandos \MakeAcronyms* e \MakeSymbols*: inserem as subdivisões da lista.
%%%% Lista de abreviaturas e siglas: [Arquivo de Entradas] (opcional)
% \MakeAcronyms[./Pre-Textual/Optionals/acronyms-entries]
%%%% Lista de símbolos: [Arquivo de Entradas] (opcional)
% \MakeSymbols[./Pre-Textual/Optionals/symbols-entries]
%%%% Glossário: [Arquivo de Entradas] (opcional)
% \MakeGlossary[./Post-Textual/Optionals/glossary-entries]
%%%% Índice remissivo
\MakeIndex%

%% Permissão de inclusão de arquivos: descomentar aqueles em edição
\includeonly{%
  ./Chapter-1/chapter-1,%% Capítulo 1 (CHKTEX 26)
  ./Chapter-2/chapter-2,%% Capítulo 2 (CHKTEX 26)
  ./Chapter-3/chapter-3,%% Capítulo 3 (CHKTEX 26)
  ./Chapter-4/chapter-4,%% Capítulo 4 (CHKTEX 26)
  ./Chapter-5/chapter-5,%% Capítulo 5 (CHKTEX 26)
  ./Chapter-6/chapter-6,%% Capítulo 5 (CHKTEX 26)
  ./Post-Textual/appendix-a,%% Apêndice A (CHKTEX 26)
  ./Post-Textual/appendix-b,%% Apêndice B (CHKTEX 26)
  ./Post-Textual/annex-a,%% Anexo A (CHKTEX 26)
  ./Post-Textual/annex-b,%% Anexo B (CHKTEX 26)
}

%% Arquivo(s) de referências
\addbibresource{./Post-Textual/references.bib}

%% Início do documento
\begin{document}%% Não comentar ou remover

%% Capa (automática)
%% Para inserir planos de fundo na capa e em uma contracapa:
%% a) selecionar a opção do pacote utfpr-thesis: CoverBG = On;
%% b) atribuir nas informações do documento argumentos válidos para:
%%    - grau acadêmico (\AcademicDegreeOption[Número]);
%%    - campus (\Campus{Cidade}).

%% Elementos pré-textuais (frontmatter)
%% Editar o {Arquivo} para alterar: folha de rosto, errata, folha de aprovação,
%% dedicatória, agradecimentos, epígrafe, resumo e abstract.
%%%% ELEMENTOS PRÉ-TEXTUAIS
%%
%% Parte que antecede o texto com informações que ajudam na identificação e na
%% utilização do trabalho.
%%
%% Observações:
%% 1. {Arg} argumento obrigatório de ambiente ou comando.
%% 2. [Arg] argumento opcional de ambiente ou comando.

%% Folha de rosto
%% Contém os elementos essenciais à identificação do trabalho, além de uma
%% licença Creative Commons (https://creativecommons.org/choose/).
%% Ambiente {TitlePage*}: aplica caixa alta no título em idioma secundário.
\begin{TitlePage}%% Argumentos (2):
[BY]%% Tipo de licença (BY, BY-SA, BY-ND, BY-NC, BY-NC-SA ou BY-NC-ND)
% [Texto da licença]%% Substitui o texto padrão para cada tipo de licença
%%%% Descrição do trabalho (padrão; alterar se necessário)
\DocumentTypeName\ presented as requirement to obtain the title of \StudentsTitlesList\ in \CourseName\ from \ifbool{MakeAcr}{\intldescr{UTFPR} (\intl{UTFPR})}{\UTFPRName\ (UTFPR)}.
%%%% Orientador(es) (de 1 a 3): {Número}; {Dados}
\Advisor{1}{%
   Gender   = {Female},%% Ou {Female}
   Title    = {\ProfCall\ \PhDCall},%% {\<T>Call}; <T>: Prof/PhD/DSc/MSc/Eng
   Fullname = {Simone Nasser Matos},%% Conforme o Currículo Lattes
%   Email    = {advisor1@domain},%% Opcional
%   Lattes   = {0000000000020001},%% Opcional
%   ORCID    = {0000-0000-0002-0001},%% Opcional (CHKTEX 8)
}
\Advisor{2}{
   Gender   = {Female},%% Ou {Female}
   Title    = {\ProfCall\ \PhDCall},%% {\<T>Call}; <T>: Prof/PhD/DSc/MSc/Eng
   Fullname = {Helyane Bronoski Borges},%% Conforme o Currículo Lattes
%   Email    = {advisor1@domain},%% Opcional
%   Lattes   = {0000000000020001},%% Opcional
%   ORCID    = {0000-0000-0002-0001},%% Opcional (CHKTEX 8)
}
%%%% Ficha catalográfica (somente para Teses e Dissertações em catálogo físico)
%%%% [Arg-1]: local (pasta) do PDF (./Pre-Textual/Extras/ por padrão).
%%%% {Arg-2}: nome do PDF (modelo em ./Pre-Textual/Extras/).
% \IndexCardPDF{doc-index-card.pdf}
\end{TitlePage}

%% Errata (elemento opcional; editar o {Arquivo} para alterar)
% \input{./Pre-Textual/Optionals/errata}

%% Folha de aprovação
%% Contém os elementos essenciais à aprovação do trabalho (sem as assinaturas).
%%%% Opção 1: gerada por meio do pacote utfpr-thesis.
%%%% Ambiente {ApprovalPage*}: insere a titulação após o nome do membro.
\begin{ApprovalPage}%% Argumentos (4):
% [brazilian]%% Idioma original ou primário (brazilian ou english)
{Month Day, Year}%{August 14, \YearNum}%% Data de aprovação (dia, mês por extenso e ano)
{14/08/\YearNum}%% Data de aprovação (forma abreviada; mestrado e doutorado)
{\linewidth}%% Largura de linha de assinatura (graduação e especialização)
%%%%%% Descrição do trabalho (padrão; alterar se necessário)
\DocumentTypeName\ presented as requirement to obtain the title of \StudentsTitlesList\ em \CourseName\ da \ifbool{MakeAcr}{\intldescr{UTFPR} (\intl{UTFPR})}{\UTFPRName\ (UTFPR)}.
%%%%%% Membro(s) da banca (de 3 a 6): {Número}; {Dados}
\Member{1}{%
  Gender      = {Female},%% Ou {Female}
  Title       = {\ProfCall\ \PhDCall},%% {\<T>Call}; <T>: Prof/PhD/DSc/MSc/Eng
  Fullname    = {Simone Nasser Matos (Advisor)},%% Conforme o Currículo Lattes
  Institution = {\UTFPRName},%% Nome completo e por extenso
}
\Member{2}{%
  Gender      = {Female},%% Ou {Male}
  Title       = {\ProfCall\ \PhDCall},%% {\<T>Call}; <T>: Prof/PhD/DSc/MSc/Eng
  Fullname    = {Leticia Mara Peres},%% Conforme o Currículo Lattes
  Institution = {Federal University of Paraná},%% Nome completo e por extenso
}
\Member{3}{%
  Gender      = {Female},%% Ou {Male}
  Title       = {\ProfCall\ \PhDCall},%% {\<T>Call}; <T>: Prof/PhD/DSc/MSc/Eng
  Fullname    = {Simone do Rocio Senger de Souza},%% Conforme o Currículo Lattes
  Institution = {University of São Paulo},%% Nome completo e por extenso
}
\end{ApprovalPage}
%%%% Opção 2: gerada a partir do Sistema Acadêmico ou da secretaria.
%%%% [Arg-1]: local (pasta) do PDF (./Pre-Textual/Extras/ por padrão).
%%%% {Arg-2}: nome do PDF (modelos em ./Pre-Textual/Extras/).
% \ApprovalPagePDF{doc-approval-page.pdf}

%% Dedicatória (elemento opcional; editar o {Arquivo} para alterar)
% \input{./Pre-Textual/Optionals/dedication}

%% Agradecimentos (elemento opcional; editar o {Arquivo} para alterar)
\begin{Acknowledgments}
I extend my heartfelt gratitude to my incredible parents and brother, whose unwavering support in every moment of need and relentless encouragement has been my steadfast anchor, propelling me forward in my academic journey.

I am profoundly grateful to my advisor, Prof. Dr. Simone Nasser Matos, whose boundless patience, unwavering faith in my potential, and constant support have been integral pillars at every stage of this work. 

My deep appreciation goes to Prof. Dr. Helyane B. Borges, whose invaluable contributions have significantly aided in the fruition of this project.

To my dear friends and my beloved girlfriend, your unwavering companionship and motivation throughout this challenging project have been indispensable. 
\end{Acknowledgments}


%% Epígrafe (elemento opcional; editar o {Arquivo} para alterar)
% \input{./Pre-Textual/Optionals/epigraph}

%% Resumo
%% Apresentação concisa dos pontos relevantes de um texto, fornecendo uma visão
%% rápida e clara do conteúdo e das conclusões do trabalho.
%% Ambiente {Abstract*}: insere a autorreferência do documento.
%%%% Estilo de fonte da chamada das palavras-chave (opcional)
% \KeywordsCallFormat{\bfseries}%% Texto normal por padrão
%%%% Palavras-chave (de 3 a 6): {Número}; {Em Português}; {In English}
\Keyword{1}{refatoração de software}{software refactoring}
\Keyword{2}{padrões de projetos}{design patterns}
\Keyword{3}{microserviços}{microservices}
\Keyword{4}{processo semiautomatizado}{semi-automated process}
%%%% Em língua vernácula (idioma primário)
\begin{Abstract}[brazilian]%% Idioma (brazilian ou english)
Refatoração é um meio de melhorar o código-fonte sem alterar a sua funcionalidade, removendo code smells e tornando-o mais flexível e legível. Dentre as técnicas de refatoração, existe a refatoração por padrões de projetos que permite criar um código com maior qualidade em relação a atributos como reusabilidade, flexibilidade, entre outros. A ferramenta Refactoring and Measurement Tool (RMT) foi criada em sua primeira versão pelo Laboratório de Engenharia de Software e Inteligência Computacional (LESIC) e é capaz de ler um projeto escrito em linguagem java e detectar e aplicar padrões de projeto em linguagem Java. Para isso, ela contém a implementação de três métodos da literatura capazes de detectar e aplicar padrões de projeto. A dificuldade da primeira versão da RMT está relacionada em se incorporar em sua arquitetura uma quantidade maior de métodos. Este trabalho realizou a refatoração de código e de sua arquitetura. O processo de melhoramento da primeira versão da RMT abrangeu as fases de: análise, restruturação, testes, refatoração e avaliação. A arquitetura da RMT foi modificada usando microserviços assíncronos e nativos na nuvem para melhorar a performance, disponibilidade e escalabilidade, desacoplando as responsabilidades. Como resultado criou-se a versão da RMT 2.0 que contém um escalonamento horizontal para melhorar a performance sobre demanda, combinando com aplicações em nuvem. Os testes criados facilitam aos desenvolvedores realizarem modificações para estender a ferramenta. O código-fonte foi simplificado para trazer melhor performance a ferramenta e melhorar a experiência do desenvolvedor que deseja contribuir com a ferramenta. O processo de execução local da ferramenta foi alterado, trazendo melhora na facilidade do mecanismo de execução, podendo ser executada com containers. A análise dos resultados evidencia uma redução de 63.64\% de tempo de execução na refatoração dos projetos testados em relação a ferramenta original.

\end{Abstract}
%%%% Em língua estrangeira (idioma secundário; para divulgação internacional)
\begin{Abstract}[english]%% Idioma (brazilian ou english)
Refactoring enhances the integrity of the source code without altering its functionality, eliminating code smells while improving its flexibility and readability. Among the various refactoring techniques, using design patterns facilitates the development of higher-quality code with enhanced attributes such as reusability and flexibility. The Refactoring and Measurement Tool (RMT) was initially developed by the Software Engineering and Computational Intelligence Laboratory (LESIC) and is capable of parsing Java projects to detect and implement design patterns. This functionality is achieved by integrating three established methodologies from the literature. The primary challenge of the first version of RMT is its limited capacity to incorporate additional methodologies into its architectural framework. This work has done a comprehensive refactoring of both the codebase and the architecture. The enhancements to the initial version of RMT were executed in phases that included analysis, restructuring, testing, refactoring, and evaluation. The RMT architecture was reengineered using asynchronous, cloud-native microservices to boost performance, availability, and scalability while segregating responsibilities. Consequently, RMT 2.0 was developed, featuring horizontal scaling to meet the performance demands associated with cloud integration. The developed testing facilitates developer modifications to extend the tool's feature set.  The source code has been optimized to enhance the tool's performance, thereby improving the development experience for contributors. The tool's local execution process has been modified to streamline the execution mechanism by allowing container-based deployment. Empirical analysis of the results indicates a 63.64\% reduction in execution time during project refactoring compared to the original tool.
\end{Abstract}


%% Lista de algoritmos (elemento opcional)
% \PrintFloatsList{algorithm}

%% Lista de ilustrações (elemento opcional; lista geral)
% \PrintIllustrationsList{figure, flowchart, photograph, graph, tabframed}
%%%% Lista de figuras (usar a partir de 3 itens e remover da lista geral)
\PrintFloatsList{figure}
%%%% Lista de fluxogramas (usar a partir de 3 itens e remover da lista geral)
% \PrintFloatsList{flowchart}
%%%% Lista de fotografias (usar a partir de 3 itens e remover da lista geral)
% \PrintFloatsList{photograph}
%%%% Lista de gráficos (usar a partir de 3 itens e remover da lista geral)
% \PrintFloatsList{graph}
%%%% Lista de quadros (usar a partir de 3 itens e remover da lista geral)
% \PrintFloatsList{tabframed}

%% Lista de tabelas (elemento opcional)
\PrintFloatsList{table}

%% Lista de abreviaturas e siglas (elemento opcional)
%%%% Opção 1: makeindex; conforme o [Arquivo de Entradas] em \MakeAcronyms.
% \PrintAcronymsList%
%%%% Opção 2: manual; editar o {Arquivo} para alterar.
% \input{./Pre-Textual/Optionals/acronyms-list}

%% Lista de símbolos (elemento opcional)
%%%% Opção 1: makeindex; conforme o [Arquivo de Entradas] em \MakeSymbols.
%%%% Comando \PrintSymbolsList*: insere a unidade (entre []) na margem direira.
% \PrintSymbolsList%
%%%% Opção 2: manual; editar o {Arquivo} para alterar.
% \input{./Pre-Textual/Optionals/symbols-list}

%% Sumário
%% Comando \PrintSummary*: remove o espaçamento entre partes e entre capítulos.
\PrintSummary%

%% Formatação de elementos textuais (mainmatter)
\Textual%% Não comentar ou remover

%% Parte 1 (elemento opcional; grupo de capítulos)
% \part{Introdução}%
% \label{part:intro}

%% Capítulo 1
\chapter{Introduction}%
\label{chpt-intro}
The definition of refactoring is to modify a software code without changing its external behavior to improve its internal structure. It is applied carefully to minimize the possibility of introducing bugs to the already written code. The refactoring does not depend on whether the code had a good design or if the software had no plan because, with refactoring, we can transform it into an excellent structure code \cite{fowler2018refactoring}.

Different kinds of refactoring exist, such as refactoring by techniques \cite{fowler2018refactoring} and refactoring with design patterns \cite{kerievsky2005refactoring}. Methods based on refactoring change the structure of the code to make it clear to understand and improve maintenance by removing code smells. On the other hand, refactoring based on design patterns improves the code structure by inserting patterns into the code.

When designing systems, multiple choices are available regarding architecture, like queues, REST and RPC requests, and so on. Microservices can combine those technologies to create large and complex architectures, as it has advantages over monolith architecture, such as fault-tolerance, organization, and development velocity \cite{microservices-comuni}.

To deploy such architecture, a cloud is an excellent option for its native tools, such as serverless computing, container managers, databases, and many other tools \cite{balalaie2016}.

The RMT is built on three services: detection service, metrics service, and intermediary service; its architecture communicates with REST requests over the intermediary service responsible for knowing all the services and making the load balance; that architecture can fit as microservices.

This work proposes refactoring the RMT architecture, bringing its design to a cloud-native approach with AWS working with offered services such as Elasticache for caching and storage, Fargate for container management, SQS as queue broker, and S3 for file storage. The refactor also includes a change from the synchronous approach to asynchronous by making the services communicate by queues and removing the responsibility of managing the requests from the client to the service itself. Besides the architecture, the flaws spotted on the tool will be fixed.

\section{Justification}
The asynchronous microservices allow services to operate independently, enabling them to scale horizontally without affecting other services; they can tolerate failures better than synchronous architectures, as they don't depend on each other's availability or response time. A failed service or request doesn't block the entire system; the different services can operate without interruption \cite{microservices-comuni}.

Asynchronous microservices can perform better than synchronous architectures by reducing latency and increasing throughput. Asynchronous processing allows the system to handle more requests concurrently and more efficiently. Microservices architecture enables greater flexibility in technology choices and enables services to be written in different programming languages, frameworks, and libraries. This architecture also allows for better integration with third-party services and APIs \cite{larrucea2018}.

Therefore, refactoring the tool will give the end user a better experience as it will be faster and more reliable than before.

% Porque os micro serviços vão ajudar na ferramenta para suportar que novos metodos sejam impletados.
\section{Objective}
Refactor the RMT to a cloud-native and asynchronous architecture to make the tool scalable, faster, and easy to use, resulting in a better experience for the end user and fixing the spotted errors on the tools.

The specific objectives are I) build an async architecture and cloud-native; II) refactor the code to work with threads to take advantage; III) refactor the java version and framework; IV) demonstrate a comparison of the refactored tool with the existing one; V) create unit tests; VI) create integration tests.

% Obejetivo geral e especifico
\section{Methodology}
This proposal aimed to refactor the architecture using a software refactoring approach involving design patterns and source code evaluation. As systematic reviews had already been conducted by \textcite{beluzzo2018abordagem} for refactoring integrating design patterns, a forward snowballing technique was utilized to search for new papers on the refactoring field not exclusively on refactoring for design patterns. Through this process, it was discovered that there were no new works on this topic, confirming the need to improve a unique tool.

The proposed approach can be done by first creating the cloud infrastructure, refactoring the project to Java 17 to add support for the last version of the Spring boot framework, adding support to the AWS SDK, and refactoring the code to communicate with all the new technology. It has to be done starting with the intermediary service, then the detection service, and finally, the metrics service.
% Descrever o que foi utilizado para chegar na solução proposta
\section{Work Organization}

This work is divided into four chapters, presenting the proposal for refactoring the RMT architecture to an async tool. The \cref{cap-background} talks about refactoring, the RMT tool, and microservices.

On \cref{chap-state} details state of the art with a forward snowballing about refactoring, tools, and methods.

On \cref{chapt-proposta} describes the proposal of refactoring the RMT, explaining how it will work and the advantages of its refactoring.

%% Parte 2 (elemento opcional; grupo de capítulos)
% \part{Desenvolvimento}%
% \label{part:dev}

%% Capítulo 2
\chapter{Theoretical Background}
\label{cap-background}

As described in \textcite{fowler2018refactoring}, refactoring is a design improvement after writing a code. Refactoring opportunities may arise throughout the project's development regardless of how the project was designed, especially in systems developed by teams.

Refactoring seeks to remove code smells encountered in software development; when software becomes mature and evolves, two conflicts arise: i) the software needs to fulfill all requirements, and ii) The software's reusability. Implementing new functionality in software without refactoring is possible, but it will eventually take great effort \cite{Gamma2009}.

Refactoring can facilitate the implementation of new requirements, modularize overloaded classes, and find methods and unnecessary classes. Different types of refactoring exist, such as techniques or methods based on design patterns. \textcite{fowler2018refactoring} is one of the most known authors. He created a catalog separating the refactorings into seven (7) groups and making it available for consultation. Fowler's online catalog encompasses the refactorings described in his two books: Refactoring: Improving the Design of Existing Code \cite{fowler2018refactoring} and Refactoring: Ruby Edition \cite{fields2009refactoring}. 

Refactoring by designing pattern-based methods examines the source code for insertion points to apply patterns. This refactoring aims mainly at maintainability, readability, and reusability. Several authors propose their pattern-based refactoring methods, such as \textcite{cinneide2000automated}, \textcite{Gamma2009}, and \textcite{ouni2017more}.

This chapter has information about the research background. \Cref{sec-importance} discuss the importance of refactoring. \Cref{sec-methods} describes the application of refactoring. \Cref{sec-tools} exemplifies the refactoring methods and the explanations of its functionalities and tools. \Cref{sub-rmt} talks about the RMT functionality. \Cref{sub-architecture} describes the architecture of the tool. \Cref{subsub-internal} talks about the internal functionality of RMT. \Cref{sub-usage} shows how to use the tool. \Cref{subsub-limitation} discusses the limitations encountered using the tool. \Cref{sec-microservices} explain about microservices. \Cref{sec2-remarks} includes the final considerations.

\section{Importance of Software Refactoring}
\label{sec-importance}
\textcite{fowler2018refactoring} describes that the first step in refactoring is to create automated tests, such as unit tests. The tests help to avoid creating bugs by guaranteeing that the behavior does not change, as any code change could also change the code behavior. Humans may make mistakes and cause serious problems in production. In system production, maintainability is a significant factor that influences costs, and one of the proposed solutions to reduce it is to increase software quality. 

Refactoring can be applied to the development process to increase software quality, forcing the programmer to find bad smells (a characteristic of the code that indicates an issue) \cite{Wilking2007}. 
In the study of \textcite{szHoke2017empirical}, six systems were analyzed to verify the effectiveness of software refactoring aimed at maintainability. To obtain data before refactoring, \textcite{szHoke2017empirical} analyzed all systems using a code analyzer called the SourcerMeter tool based on the Columbus tool \cite{ferenc2002}. Refactorings were performed manually, and developers were informed of all the data and the list of problematic code fragments. 

In the study \textcite{szHoke2017empirical}, developers analyzed 2.5 million lines and performed 732 code revisions, of which 315 were refactorings and 1,273 refactoring operations. After finishing the analysis, \textcite{szHoke2017empirical} found that refactoring improved maintainability and that the tool improved 5 of the six systems tested. In conclusion, the study shows that refactoring improves maintainability, which helps to meet future software requirements.

\subsection{Methods to Apply refactoring}
\label{sub-methods}
As mentioned, we can apply refactoring techniques and methods based on design patterns. Refactoring by techniques is common and can be found in leading IDEs, such as Eclipse and IntelliJ, which provide automatic refactorings based on the model created by \textcite{fowler2018refactoring}. 

Several authors propose software refactoring methods applying patterns, each performed differently in refactorings using pattern-based design methods. Some of these authors are \textcite{liu2014automated}, \textcite{zafeiris2017automated}, \textcite{cinneide2000automated}, and \cite{ouni2017more}. 

These refactoring methods seek to find the point of insertion of the pattern in the source code. The technique searches for source code to find where to implement design patterns to improve code maintainability, readability, and reusability. 

Unlike technical refactoring, design pattern-based refactoring looks for parts of the code where a design pattern can be applied. Using a design pattern is very important because it makes the evolution of the software more natural and does not create problems when implementing new requirements. Several project patterns have different functionality, such as the factory method, proxy, observer, adapter, visitor, builder, and others \cite{Gamma2009}). 

As described in the mapping performed by \textcite{beluzzo2018abordagem}, several authors propose methods for refactorings based on design patterns. For example, in the Moore method and Minipatterns and Minitransformations, there is a tool for each implementation process. If the developer wants to refactor his source code, he has to run n tools, where n is the total number of methods.

The focus of refactoring by design patterns differs from refactoring by techniques. Refactoring by design patterns tries to find a way to make a system easy to "update." Software that is difficult to update can cost the company a lot in value and the need for new clients and competitiveness with other companies \cite{cinneide2000automated}.

The next section presents methods for detecting and applying design patterns described in the literature.

\subsection{Methods and Tools for Refactoring Based on Design Patterns}
\label{sec-tools}

As mentioned, \textcite{beluzzo2018abordagem} identified methods for detecting and involving design patterns as described in \Cref{tab-articles}.

\begin{longtable}{e{}@{},{}@{}p{3cm}p{8cm}cc@{}}
%% Cabeçalho da primeira página
\caption{Articles with design patterns methods}
\label{tab-articles}                                              \\[\belowcaptionskip]
\multicolumn{5}{@{}r@{}}{\textbf{(continue)}}                 \\[\belowcaptionskip]
\toprule%
\multicolumn{1}{@{}c}{\textbf{Key}}             &
\multicolumn{1}{c}{\textbf{Author}}             &
\multicolumn{1}{c}{\textbf{Title}}              &
\multicolumn{1}{c}{\textbf{Year}}               &
\multicolumn{1}{c@{}}{\textbf{Has Tool}}        \\
\midrule%
\endfirsthead%
%% Cabeçalho das páginas (exceto primeira e última)
\caption[]{Articles with design patterns methods}             \\[\belowcaptionskip]
\multicolumn{5}{@{}r@{}}{\textbf{(continuation)}}             \\[\belowcaptionskip]
\toprule%
\multicolumn{1}{@{}c}{\textbf{Key}}             &
\multicolumn{1}{c}{\textbf{Author}}             &
\multicolumn{1}{c}{\textbf{Title}}              &
\multicolumn{1}{c}{\textbf{Year}}               &
\multicolumn{1}{c@{}}{\textbf{Has Tool}}        \\
\midrule%
\endhead%
%% Cabeçalho da última página
\caption[]{Articles with design patterns methods}             \\[\belowcaptionskip]
\multicolumn{5}{@{}r@{}}{\textbf{(conclusion)}}               \\[\belowcaptionskip]
\toprule%
\multicolumn{1}{@{}c}{\textbf{Key}}             &
\multicolumn{1}{c}{\textbf{Author}}             &
\multicolumn{1}{c}{\textbf{Title}}              &
\multicolumn{1}{c}{\textbf{Year}}               &
\multicolumn{1}{c@{}}{\textbf{Has Tool}}        \\
\midrule%
\endlasthead%
%% Rodapé da última página
\bottomrule%
\LTSourceOrNote{adapted from \textcite{beluzzo2018abordagem}}           \\
\endlastfoot% 
A1  & \citeauthor*{GAITANI201533}            & Automated refactoring to the Null Object design pattern                                                             & 2015 & Yes      \\
A2  & \citeauthor*{CHRISTOPOULOU20121201}    & Automated refactoring to the Strategy design pattern                                                                & 2012 & Yes      \\
A3  & \citeauthor*{zafeiris2017automated}           & Automated refactoring of super-class method invocations to the Template Method design pattern                       & 2017 & Yes      \\
A4  & \citeauthor*{CINNEIDE2015}             & A multi-objective refactoring approach to introduce design patterns and fix anti-patterns                           & 2015 & Yes      \\
A5  & \citeauthor*{cinneide2001automated}    & Automated application of design patterns: a refactoring approach                                                    & 2001 & Yes      \\
A6  & \citeauthor*{cinneide792644}           & A Methodology for the automated introduction of design patterns                                                     & 1999 & Yes      \\
A7  & \citeauthor*{mens972774}               & A declarative evolution framework for object-oriented design patterns                                               & 2001 & Yes      \\
A8  & \citeauthor*{sang1183003}              & An automated refactoring approach to design pattern-based program transformations in Java programs                  & 2002 & Yes      \\
A9  & \citeauthor*{Cinneide602499}           & Automated Software Evolution Towards Design Patterns                                                                & 2001 & Yes      \\
A10 & \citeauthor*{cinneide337612}           & Automated refactoring to introduce design patterns                                                                  & 2000 & Yes      \\
A11 & \citeauthor*{Liu2014}                  & Automated pattern-directed refactoring for complex conditional statements                                           & 2014 & No       \\
A12 & \citeauthor*{ram2004detecting}         & Detecting Intent Aspects from Code to Apply Design Patterns in Refactoring: An Approach Towards a Refactoring Tool  & 2004 & Yes      \\
A13 & \citeauthor*{hotta6178876}             & Identifying, tailoring and suggesting form template method refactoring Opportunities with program dependence graph. & 2012 & Yes      \\
A14 & \citeauthor*{rajesh1013988}            & JIAD: A tool to infer design patterns in refactoring                                                                & 2004 & Yes      \\
A15 & \citeauthor*{ouni2017more}             & MORE: A multi-objective refactoring recommendation approach to introducing design patterns and fixing code smells   & 2017 & Yes      \\
A16 & \citeauthor*{eden632834}               & Precise specification and automatic application of design patterns                                                  & 1997 & No       \\
A17 & \citeauthor*{kerievsky2005refactoring} & Refactoring to Patterns                                                                                             & 2008 & No       \\
A18 & \citeauthor*{kim7332467}               & Scripting parametric refactorings in java to retrofit design patterns                                               & 2015 & No       \\
A19 & \citeauthor*{kim2014scripting}         & Scripting Refactorings in Java to Introduce Design Patterns                                                         & 2014 & Yes      \\
A20 & \citeauthor*{juillerat4362900}         & Toward an implementation of the ”Form Template Method” Refactoring                                                  & 2007 & Yes      \\
A21 & \citeauthor*{ajouli6619484}            & Transformations between composite and visitor implementations in Java                                               & 2013 & Yes     
\end{longtable}
\FloatBarrier

Each article has methods to recognize design pattern insertions, but not every article has a tool to apply it, as described in \Cref{tab-articles}. \textcite{beluzzo2018abordagem} searched for the articles until 2018, when his work was published. A new search was conducted with the same search strings and libraries from 2019 to 2024; no new work was found.

Some steps should be taken to refactor the code and apply transformations, as refactoring is not a straightforward technique and may vary by who is using it. That is why the authors have taken different approaches to refactoring listed in \Cref{tab-refactoring}.

\begin{table}[!htbp]
\caption{Refactoring Types}%
\label{tab-refactoring}
\begin{tabularx}{\textwidth}{c c}
\toprule%
\textbf{Keys}        & \textbf{Types}   \\      
\midrule%
A15, A10, A9, A6, A5 & MiniTransformations                                        \\
A2, A11              & Conditional Expression Refactorings                        \\
A12, A14             & Intent Aspects Refactoring*                                \\
A18, A19             & Reflective Refactorings                                    \\
A4, A15, A21         & Composite to Visitor and Visitor to Composite Refactorings \\
A4, A7, A15          & Role based Refactorings                                    \\
\bottomrule%
\end{tabularx}
\SourceOrNote{adapted from \textcite{beluzzo2018abordagem}}
\end{table}


A refactoring type is applied to the code for every work, as described in \Cref{tab-refactoring} and listed in \Cref{tab-refactoring}. Minitransformations are minitransformations that are simple modifications to the code. By grouping these transformations in the right way, a design pattern is created \cite{cinneide2001automated}.

Reflective refactoring is based on code reflections, a technique that allows the code to be changed on runtime. It is good to emphasize that there are no pre-refactoring implementations with reflections; they are a method to implement the refactorings \cite{beluzzo2018abordagem}. 

Role-based refactoring is defined by a group of roles corresponding to the pattern participants and applying those roles to the refactor, obtaining the design pattern \cite{mens972774}. 

Intent aspects are software susceptible to pattern refactoring, formed by a group of rules for each pattern\cite{ram2004detecting}. Composite to Visitor and Visitor to Composite focus only on the transition between each pattern \cite{beluzzo2018abordagem}. Finally, Conditional Expression Refactorings focus on the code's conditional expressions and conditional branches to refactor \cite{CHRISTOPOULOU20121201}.

As shown in \Cref{tab-articles}, most articles have tools developed to apply your specific method, working in an automatic or semi-automatic manner.

\begin{tabframed}[!htbp]
\caption{Developed tools}%
\label{tab-tools}
\begin{tabular}{|l|l|}
\toprule%
\multicolumn{1}{|@{}c|}{\textbf{Keys}}  &
\multicolumn{1}{c@{}|}{\textbf{Tools}} \\
\midrule%
A2, A3         & JDeodorant                   \\
A4             & MORE                         \\
A6, A7, A8, A9 & Prototype\textbf{*}          \\
A5, A10        & Design Pattern Tool          \\
A13            & Creios                       \\
A12, A15       & JIAD                         \\
A15            & MORE                         \\
A11, A19, A20  & Eclipse Plugin\textbf{*}     \\
A21            & JHotDraw                     \\
\bottomrule%
\end{tabular}
\SourceOrNote{adapted from \textcite{beluzzo2018abordagem}}
\end{tabframed}
\FloatBarrier

In \Cref{tab-tools}, the developed tools are listed by name; the row marked with an asterisk means that the tool is a prototype or an eclipse plugin; the other articles were created or updated of specific tools\cite{beluzzo2018abordagem}.

The tool creation is necessary because manual refactoring, the process some developers use, can introduce errors to the code and even change the system's functionality. This problem occurs because developers are not prepared to perform a refactoring \cite{ge2012reconciling}.

Manual refactoring requires reviewing the entire code and can take time. For this reason, tools facilitate the application of refactorings, which can identify bad smells or identify and refactor the source code, as created by \textcite{beluzzo2018abordagem}.

\textcite{beluzzo2018abordagem} performed a mapping to find refactoring techniques, and several tools have already been developed to perform refactorings. Such as Elbereth \cite{korman1998elbereth}, which automates some code extraction refactorings, encompassing method extraction (Extract Method), abstract superclasses, replacement of an existing class, and addition of a new subclass.

To improve the interaction with the user for the application of refactoring, \cite{murphy2008breaking} created a tool that is fast, resistant to errors, and pleasant to use. To make the tool enjoyable, they implemented markings in the code produced by rectangles, arrows, and coloring parts of the code. These three markings show the user cleanly where and what the problem is so that refactorings can be applied.
Some tools contain a graphical interface and iteration with the user, such as \textcite{rani2014detection}, where the intent is to detect bad smells. This tool was developed in C\# and finds terrible smells such as the long method, large class, lazy class, and comment lines. Discovering smells in code written in Java and .net. 

Many code refactoring tools are based on the catalog of refactorings created by \textcite{fowler2018refactoring}, as in the integrated development environment (IDE) (e.g., Eclipse, Netbeans, and IntelliJ). However, some authors utilize design pattern-based refactoring as in \cref{tab-tools}. Authors, such as \textcite{cinneide2000automated} and \textcite{CHRISTOPOULOU20121201} agree that automated refactoring reduces application time and can be more beneficial, as it allows less experienced developers to apply refactorings. They were making tools for this reason.

\textcite{murphy2008breaking}, together with eight software developers with different work experiences, used the GostFactor tool to assist in refactoring. They observed that this process decreased errors in the refactoring application by 23.3\%.

The RMT tool, as developed by \textcite{beluzzo2018abordagem}, facilitates the seamless integration of design pattern insertion and detection methods within a unified environment. This tool engages in an interactive process with the developer, providing foresight into the inherent changes after refactoring and concurrently evaluating the effectiveness of such refactoring in enhancing code quality. \textcite{sangeetha2019empirical} 's work, although it does not explain tool implementation, employs it for experimental verification and validation of the used techniques. The RMT is designed to support the implementation of any refactoring method in the literature \cite{beluzzo2018abordagem}.

\subsection{RMT}
\label{sub-rmt}

The RMT (Refactoring and Measurement Tool) was created to identify design pattern insertion points utilizing the methodologies implemented by \textcite{beluzzo2018abordagem}. \textcite{zafeiris2017automated} and \textcite{liu2014automated} are responsible for developing the implemented refactoring techniques. The developed approaches are, respectively, the refactoring of the Template Method pattern and the refactoring of the Strategy and Factory Method patterns. The \cref{fig-activity-rmt} illustrates the flow of the RMT tool without discriminating between implementations. The figure is divided into two distinct colors, each representing a unique request (thread) on hold by the client user and, consequently, by the intermediary service.

\begin{figure}[ht!]
\SetCaptionWidth{\textwidth}
\caption{Activity diagram for RMT}
\label{fig-activity-rmt}
\fontsize{3.8}{5}\selectfont
\includesvg[width =\textwidth]{Chapter-2/Figures/activity-rmt.svg}
\SourceOrNote{Own authorship (2024)}
\end{figure}
\FloatBarrier

The use activity diagram delineates the tool's functionality for the user. The client initiates the process by uploading the project and searching for potential candidates. Upon identification of any applicable methods, the corresponding metric is computed. Within the interface, the client can select a class for refactoring, if necessary, and subsequently execute the refactoring process. Without identified candidates, the process ends without proceeding to refactoring and metric computation.

In addition to applying the refactorings, the tool uses the CK metrics extractor \cite{ck}, which encompasses the metrics of the depth of the inheritance tree (PAH), the cyclomatic complexity (CC), and the size of the program in lines of code (TPLC). Upon selecting a project selected for refactoring, the metrics are applied to demonstrate the effects of refactoring on the user source code. The available external quality attributes are maintainability, reliability, and reusability.

\subsubsection{RMT Architecture}
\label{sub-architecture}

The architecture diagram in \cref{fig-architecture} starts with user iteration within the Client App, as it is isolated and runs independently since it is a Java desktop application; the communication can be only made with the Intermediary Service, which provides the communication between the user layer (Client) and the processing layer (Services). The Intermediary Service uses Hypertext Transfer Protocol (HTTP) requests to communicate to the subsequent Metrics and the Detection Service.

\textcite{beluzzo2018abordagem} model also proposes separating the services by region, which may offer better availability. This mitigation is accomplished by distributing services across multiple geographic regions (including distinct cities, states, and countries), thereby ensuring continuity of service and preventing total system failure in the event of a regional outage. The database is a global service that can be accessed from any region. \Cref{fig-architecture} represents the regional divisions of application services.

\begin{figure}[ht!]
\SetCaptionWidth{\textwidth}
\caption{RMT architecture diagram}
\label{fig-architecture}
\includegraphics[width =\textwidth]{Chapter-2/Figures/schema.png}
\SourceOrNote{\textcite{beluzzo2018abordagem}}
\end{figure}
\FloatBarrier

The intermediary service functions as a service registry, possessing detailed knowledge of every service across all regions and strategically selecting the optimal server for communication. At a high level, upon receiving a project from the user, the intermediary service engages with the detection service, which subsequently provides the refactoring candidates. Following this, the intermediary service requests the detection service to perform the refactoring on the identified candidates. Upon receiving the responses from the refactored candidates, the metrics service is invoked to compute the quality attributes. Ultimately, the intermediary service aggregates all pertinent information and formulates a response for the user.

\textcite{beluzzo2018abordagem} attempted to conceptualize an extensible tool, facilitating new implementations without experiencing substantial obstacles. However, he did not delineate a concrete process for extension.

\subsubsection{Internal structure}
\label{subsub-internal}

The architecture of a computational system is critical for its proper functionality; however, most intensive processing operations are executed within the services. An optimally designed service will operate efficiently with minimal memory and CPU resources, ensuring compatibility with small-scale computer instances. The methodology for selecting candidates for refactoring using RMT is illustrated in \Cref{fig-candidates}.

\begin{figure}[ht!]
\SetCaptionWidth{\textwidth}
\caption{Sequence Diagram to Find Refactoring Candidates}
\label{fig-candidates}
\includegraphics[width=160mm]{Chapter-2/Figures/candidates.png}
\SourceOrNote{Own authorship (2023)}
\end{figure}
\FloatBarrier

As described, the intermediary service communicates with the detection service after saving the project to the database and sends the project ID, as in Step 1.

The detection service receives an HTTP request on the \texttt{DetectionBoundary} class and, with the project ID, calls the \texttt{extractCandiates} method. The controller process occurs in Step 1.1.

The id is received by the \texttt{ExtractManager} class, which is the interface responsible for calling the \texttt{getProject} method to retrieve the project from the repository; the actions are represented by Step 1.2.

The \texttt{ProjectRepostiory} is an interface with database access that is used to retrieve project information along with the project zip file as \texttt{InputStream} in the \texttt{Project} entity, completing Step 1.3. Step 2 sends the retrieved project to the \texttt{DataExtractionFork}.

The \texttt{DataExtractionFork} has direct access to the system's input and output (io), saving the project and retrieving it immediately afterward. The project is saved as a zip file to be parsed as a \texttt{CompilationUnit} class (class representation of abstract syntax tree) and loaded into the memory to be sent to every implementation of the DetectionMethod as in Step 2.1.

The \texttt{DetectionMethod} is an interface that abstracts the functionality of a refactoring method. It receives an instance of the \texttt{DataExtactionFork} and iterates over every file to detect the possibility of designing the pattern insertion; the process results are returned to the intermediary service, finishing Steps 3.1 to 3.5.

If any candidate was detected, the intermediary service automatically calls the service of the detection to refactor those candidates as in \Cref{fig-refactoring}.

\begin{figure}[ht!]
\SetCaptionWidth{\textwidth}
\caption{Sequence Diagram for Refactoring Candidates}
\label{fig-refactoring}
\includegraphics[width=160mm]{Chapter-2/Figures/refactoring.png}
\SourceOrNote{Own authorship (2023)}
\end{figure}
\FloatBarrier

\texttt{DetectionBoundry} class is the application controller, it recives the request to start the refactoring, within the request are the classes choose by the user to be refactored and the project ID, the information is send to the \texttt{ExtractManager} as Step 1.1.

The \texttt{ExtractManager} class has the same functionality as in \Cref{fig-candidates}, retrieving the project from the database by the \texttt{ProjectRepository} (Step 1.2); and then calling the \texttt{DataExtractionFork} class over a loop as Step 2.

 In Step 2.1, \texttt{DataExtractionFrok} has the same functionality as in \cref{fig-candidates}. However, every candidate in the project inflates the zip file and converts it to a \texttt{CompilationUnit} class to be refactored in Step 3. After refactoring has ended, the project ID is returned from 3.1 to 3.4.

\subsubsection{RMT Usage}
\label{sub-usage}

The main focus of explaining the usage of RMT is on the Client App, which is divided into three stages. The first step is to import the project, as shown in \Cref{fig-import}.

\begin{figure}[ht!]
\SetCaptionWidth{\textwidth}
\caption{Importing project on clientApp}
\label{fig-import}
\includegraphics[width =100mm]{Chapter-2/Figures/import.png}
\SourceOrNote{Own authorship (2023)}
\end{figure}
\FloatBarrier

The second step is the project evaluation, giving feedback to the user, which can analyze additional information about the candidates for refactoring, such as class name, design pattern, and metrics, as shown in \Cref{fig-choose}.

\begin{figure}[ht!]
\SetCaptionWidth{\textwidth}
\caption{Selecting refactoring candidates}
\label{fig-choose}
\includegraphics[width =\textwidth]{Chapter-2/Figures/choose.png}
\SourceOrNote{Own authorship (2023)}
\end{figure}
\FloatBarrier

In the third and last step, the user can choose the candidates to apply the refactoring after using it to a new project created and saved in a user-selected directory, as shown in \Cref{fig-refactor}.

\begin{figure}[ht!]
\SetCaptionWidth{\textwidth}
\caption{Applying refactoring to candidates}
\label{fig-refactor}
\includegraphics[width =\textwidth]{Chapter-2/Figures/refactor.png}
\SourceOrNote{Own authorship (2023)}
\end{figure}
\FloatBarrier

The metrics in the tool are expressed in percentages where the positive value shows an increase in the given quality attributes and the negative value shows a decrease. The attributes are calculated by combining the following metrics: cyclomatic complexity, depth of the inheritance tree, and lines of code. The combinations of all metrics calculate maintainability; Reliability uses the cyclomatic complexity and lines of code, and reusability uses the depth of the inheritance tree and lines of code. 

\subsubsection{RMT Limitations}
\label{subsub-limitation}
When testing the RMT, some limitations were noticed. The first limitation was the long time required to execute a project refactoring, which involved drilling down on the tool code to find the reason for the slowness. One section had strange behavior: starting a refactoring system, extracting a zip file, and parsing every file to memory to verify if it could be a refactoring candidate. The problem is that the parsing process is repeated for each file and is repeated even more when the candidate is refactored. Following the same idea, when the candidate is refactored, each refactoring step is saved on the filesystem, once again accessed, and parsed to retrieve the changes. The disk (filesystem) has expensive access and takes time; a better option is to handle all the access directly on the memory.

The blocking request issue manifests when many users concurrently access the tool, thereby saturating the available connections. Any new users will increase latency and may induce users to receive an error from the server as they have to wait until other processes are over to open space for a new connection. This problem mainly occurs because of the synchronous architecture on RMT based on an API manager (intermediary service) communicating with the other services via HTTP requests, holding the service thread until the call ends. This behavior could cause problems in other functionalities, such as the load balancer, service registry, and service discovery, which occurs because the service has no more threads to execute any code. It may crash or freeze. Since Java 8 is limited by the system threads, scaling vertically (upgrading the CPU) is the only way to support more clients. The downside of HTTP communication is that latency increases with the number of concurrent clients \cite{Cebeci2020DesignOA}. 

In the detection service, the Java Parser library \textcite{javaparser} is in versions that support only features up to Java 9 (the last release is Java 22). That was not a limitation when the tool was built but became a limitation as it ages. This limitation can be addressed by constantly updating the library.

Another limitation is related to the CK \textcite{ck} library used to calculate the metrics; the library code was copied to the tool code, making it difficult to update for new versions and take advantage of new metrics and features. 

The RMT shows two refactorings for the same design pattern and class. Therefore, the user has only the metrics to choose from, which will be applied; there are no other ways to explain the difference between the method to the user to make a better decision.

The tool currently lacks unit tests to ensure the logic implemented within the class is accurate. Such tests are instrumental in facilitating refactoring efforts, ensuring that while the code undergoes structural modifications, its intended functionality remains unchanged as long as the tests pass.

\section{Microservices Architecture}
\label{sec-microservices}
The microservices are used to create large and complex applications, as the model shown in \Cref{fig-architecture}, each application must be simple and independent. When services are connected and working together, they become a system. As discussed in \textcite{microservices-comuni}, the application is fault-tolerant and more controllable than a monolithic architecture.

There are many architectures, such as synchronous RMT and asynchronous ones. The synchronous services wait for the response before ending the process; they usually use a direct connection over Representational State Transfer (REST), Remote Procedure Call (RCP), etc. \cite{microservices-comuni}. To implement a retry on synchronous services, the caller service must handle the failure without terminating the connection.

Asynchronous services do not wait for a response; the communication is non-block, so the service does not wait for the answer to end the process; it can be achieved with queues as communication means. Some queue services are RabbitMQ, Apache Kafka, and Simple Service Queue (SQS), among others \cite{KARABEYAKSAKALLI2021111014}. Queue services have resilience-oriented characteristics, including the ability to persist unconsumed messages, thus ensuring that requests are not lost even if a service experiences downtime \cite{Cebeci2020DesignOA}.

Queue services offer mechanisms to improve system resilience by facilitating message retry attempts. One of these features is visibility timeout when a message is visible to a single consumer in the queue; if the message is not processed before the visibility timeout, it is added to the retry count and is visible to be consumed again. The queue must be configured with a retry limit to avoid an infinite retry. A notable limitation is the potential for message duplication in scenarios involving operational failures \cite{ChenScalable}.

Microservices have emerged as a hot topic within the domain of software development, garnering increased attention for their ability to significantly improve the maintenance and scalability of online services. A systematic mapping article by \textcite{Alshuqayran} carefully examines various microservice architecture initiatives, highlighting their growing importance within academic discourse. This study provides an exhaustive analysis of the deployment of microservices in various projects, illustrating practical applications.

\section{Closing Remarks}
\label{sec2-remarks}
This chapter reported on the importance of refactoring, described the methods based on design patterns and refactoring tools, and explained some aspects of service architecture.

Refactoring is essential to keep the source code free from smells and to a quality standard. The main focus of refactoring is maintainability and reusability by maintaining the code as first designed and avoiding inserting new smells.

The importance of using a software refactoring tool was addressed to get the most out of refactoring without harming the work already done, highlighting that the RMT tool is the focus of this research.
The RMT tool integrates several methods for detecting and inserting design patterns in a single environment so that the application developer can apply them to their source code without using several refactoring tools.

There are many types of software architecture, but we discuss the abilities of the async and sync applications, bringing both upsides and downsides. This work used snowballing to ensure that no other tool, such as RMT, was used, and the results are presented in the next chapter.

%% Capítulo 3
\chapter{State of Art}%
\label{chap-state}

Systematic literature review (SRL) studies are traditionally in software engineering, and software refactoring brings an excellent background for study analysis and classifications. As time passes, studies become obsolete as new articles are published monthly. A forward snowballing was proposed to update the systematic literature reviews on software refactoring.

This chapter is structured as \Cref{sec-background} describing the research methodology; \Cref{sec-methods} explaining the methods chosen to apply to the research; \Cref{sec-results} showing the results obtained in the study; \cref{sec-trends} explaining the trends in the research; \Cref{sec-cloasing-remarks} are the closing remarks.

\section{Background}
\label{sec-background}
According to \cite{bernard2006}, the snowballing technique is a nonprobabilistic sampling technique that allows the reach of hard-to-reach or little-known populations. It occurs due to its mechanism of establishing a network of relationships among the elements, creating a chain of references.

The technique has three objectives: to improve the understanding of a theme, verify the possibility of conducting a more extensive study, and develop methods to be employed in subsequent studies and phases \cite{vinuto2014}. Snowballing is used mainly for experimental purposes.

The procedure for following the Snowballing technique is described in the following steps: start set, iterations (backward snowballing, forward snowballing, and inclusion and exclusion), identification of the authors, and data extraction \cite{Wohlin2014}.

The first step of snowballing is creating a starting set of papers; a viable option is to use search engines such as Google Scholar. It is an excellent alternative to avoid bias in favor of any author \cite{Wohlin2014}.

\textcite{Kitchenham2013} chose a set of works to start the snowballing; they preferred to use two conference proceedings, the Evaluation and Assessment in Software Engineering and Empirical Software Engineering and Measurement Searching, from 2005 to mid-2012. We chose first to utilize a database search with a string and search engines as it avoids the bias of a starter set of SRL before the forward snowballing to update the fancied SLR.

The title and abstract are the elements to consider when selecting suitable candidate papers. The main goal is to avoid unrelated works, but only if they are explicitly irrelevant, and the premise is to include any paper that may be relevant. To select or remove articles, all authors must agree, and if there is any estrangement, discussions may occur until their consent is defined \textcite{Kitchenham2013}.

After establishing the initial set, the iteration step is performed by performing Backward Snowballing and Forward Snowballing, including and excluding new jobs in the collection \cite{Wohlin2014}.

Backward snowballing consists of using the list of references to identify new articles to be analyzed later on. The first step is to review the references and discard the papers that do not meet the essential research criteria. The second step is to remove the reports that have already been examined from the list. The first two steps are about extracting information from the article, and a new article should not be studied if there is not enough information in the analyzed document \cite{Wohlin2014}.

Conversely, forward snowballing consists of using the list of citations to identify new articles to be analyzed later. The first step is to review the citations in an online database and discard papers that do not meet the essential criteria. The following steps follow the same concept as forward-backward snowballing \cite{Felizardo2016}.

\section{Methods}
\label{sec-methods}
Database searches were used to start a systematic literature review update. For instance, to find a search string, the databases were chosen to search for and determine the exclusion and inclusion to decide which of the found studies are adequate to enter as a starter set for the forward snowballing.

It is challenging to assemble a suitable string, as the terminology used in software engineering is not standardized. Using a specific keyword may find a few relevant articles and even miss some suitable works; otherwise, using generic keywords may result in irrelevant articles, creating unnecessary labor \cite{Wohlin2014}.

To find the starter set of SLR to apply the forward snowballing, a database search with a specific set of keywords was determined, and the found collection of papers will be updated as a result of the process. The search string to initiate the snowballing approach is (("systematic literature review") OR ("systematic review") OR ("systematic mapping review") OR ("systematic mapping") ) AND "software refactoring" and includes all the main desired topics to search, such as SLR and systematic mapping review (SMR) with two options keywords for each to result in a more accurate search on the subject. An inclusion and exclusion criteria list was defined to select the papers returned by the search string.

The inclusion criteria are as follows:

\begin{itemize}
    \item The publishing year of the articles must be between 1992 and 2022, as \textcite{Opdyke1992} was the first person to publish with the refactoring term;
    \item Systematic literature review articles and systematic mapping reviews about software refactoring methods, frameworks, technics; or applications or development, methods or practices code smells detections;
\end{itemize}

The criteria for exclusions are as follows.

\begin{itemize}
    \item Article not written in English or Portuguese;
    \item Non-source code studies or architecture refactoring;
    \item Abstract, posters, patents, and keynotes;
\end{itemize}

The standards must be followed by reading the article's title to filter the studies using the above criteria. If it already has the information to be accepted, no more reading is necessary for inclusion; if only the title is insufficient, the abstract is thoroughly read to understand more about the paper's objectives. The whole article is read if there is still doubt about the exclusion or inclusion. \textcite{Wohlin2014} does not recommend reading the report from the beginning; instead, he argues that it is best to read the most relevant parts to make a decision.
The search engines were Google Scholar, Springer, Science Direct, ResearchGate, IEEE Xplore, and ACM. From these bases, 804 articles were divided per database: 689 from Google Scholar, 43 from Springer, 32 from Science Direct, 25 from ResearchGate, 12 from ACM, and three from IEEE Xplore, as shown in \Cref{tab-reviews}.

To apply forward snowballing, Google Scholar was chosen to look for citations, as it is an excellent way to avoid bias \cite{Wohlin2014}. All articles were submitted to one interaction with the original SRL as sed set, evidenced by \textcite{Wohlin2020} as the most cost-effective approach to update the studies. The diagram in \Cref{fig-snow} exemplifies the whole search method. 

\begin{figure}[ht!]
\SetCaptionWidth{\textwidth}
\caption{Search Method Diagram}
\label{fig-snow}
\includegraphics[width =100mm]{Chapter-3/Figures/snowballing_diagram.png}
\SourceOrNote{Own authorship (2023)}
\end{figure}
\FloatBarrier

As in the database search, snowballing inclusion and exclusion criteria must be applied in the papers that have cited the current SRL. The requirements must come from the starting set of documents \cite{Wohlin2020}. Reading every SRL, it generally finds a section explaining those accepted standards.


\section{Results}
\label{sec-results}
\Cref{tab-reviews} shows all the papers found on the database search to achieve the start set for snowballing; it contains an id composed of S representing the start and a number. The table is sorted by title and contains the total number of citations found in Google Scholar for each article and the reference.

\begin{tabframed}
\caption{Select papers from database search}
\label{tab-reviews}
\begin{tabularx}{\textwidth}{|e{}@{},{}@{}|p{9cm}|c|p{3cm}@{}|}
\toprule%
\multicolumn{1}{|@{}c|}{\textbf{Key}}        &
\multicolumn{1}{c|}{\textbf{Tile}}    &
\multicolumn{1}{c|}{\textbf{Citation  Nº}}  &
\multicolumn{1}{c@{}|}{\textbf{Ref}}        \\
\midrule% 
S1  & 30 Years of Software Refactoring Research:A Systematic Literature Review                           & 13          & \citeauthor*{Abid2020}                               \\
S2  & A Literature Review on Code Smells and Refactoring                                                 & 17          & \citeauthor*{Ruben2010}                              \\
S3  & A Systematic Literature Review on Software Refactoring                                             & 0           & \citeauthor*{Elhazzat2020}                           \\
S4  & A Systematic Literature Review on Software-refactoring Techniques, Challenges, and Practices       & 0           & \citeauthor*{Akhtar2022}                             \\
S5  & A systematic literature review: Refactoring for disclosing code smells in object oriented software & 74          & \citeauthor*{Singh2018b}                              \\
S6  & A systematic literature survey of software metrics, code smells and refactoring techniques         & 15          & \citeauthor*{Agnihotri2020}                          \\
S7  & A systematic mapping of literature on software refactoring tools                                   & 3           & \citeauthor*{Tavares2018}                            \\
S8  & A systematic review on search-based refactoring                                                    & 78          & \citeauthor*{Mariani2017}                            \\
S9  & Automatic software refactoring: a systematic literature review                                     & 40          & \citeauthor*{Baqais2020}                             \\
S10 & Classification and Summarization of Software Refactoring Researches: A Literature Review Approach  & 4           & \citeauthor*{Abebe2014b}                              \\
S11 & Code Smells and Refactoring: A Tertiary Systematic Review of Challenges and Observations           & 47          & \citeauthor*{Lacerda2020}                            \\
S12 & Multi-Objective Optimization Techniques for Software Refactoring: A Systematic Literature Review   & 1           & \citeauthor*{Rafique2019}                            \\
S13 & On preserving the behavior in software refactoring: A systematic mapping study                     & 15          & \citeauthor*{AlOmar2021}                             \\
S14 & Trends, opportunities and challenges of software refactoring: A systematic literature review       & 42          & \citeauthor*{Abebe2014a}                              \\
S15 & Why , How , and When Refactorings are ( NOT ) Applied : A Systematic Literature Review             & 0           & \citeauthor*{Buriakovskyi2018}                                \\
\bottomrule%
\end{tabularx}
\SourceOrNote{Own authorship (2023)}           
\end{tabframed}
\FloatBarrier

The most crucial part of updating the SRLs in the acceptance criteria is that their interpretation will decide if a cited paper has the qualities to be included in its update. Most studies have an explicit acceptance criterion besides S3, S7, S10, and S12. Although S12 has no detailed parameters, the questions are well-defined enough to use as a selection method for snowballing; unfortunately, the paper has no citations. For S10, the criteria to include in the update were the classification made from the SLR-selected articles. However, finding an alternative to filter the citations became less secure in fitting the original article because the author did not have a particular sentiment. 

Multiple articles did not include some styles of publications, such as abstracts, posters, and keynotes for S8, S9, and S13, gray literature for S1, S8, and S9, doctoral symposiums, books, and theses for S8 and S9. In the gray literature, it may not be the most reasonable guideline to follow because, as shown in articles S1 and S11, the first time worked with the term refactoring was in 1992 by \textcite{Opdyke1992}, which was a Ph.D. thesis. According to \textcite{Buriakovskyi2018}, only after the published book \textcite{fowler2018refactoring} did the active research on the practical use of refactoring begin. This would exclude both studies from S8 and S9.

The number of studies selected and filtered for each article was quite different; after removing duplicates, the final result was six articles for S1, 4 for S2, 13 for S5, 4 for S6, 16 for S8, 5 for S9, 1 for S10, 15 for S11, and 1 for S13. Sixty-five possible papers will be included in future SLR updates as the analyses cover each article's approval criteria.

 \Cref{tab-snow} shows all the selected papers from applying the forward snowballing and following the acceptance criterion from all the SLRs. The first row has the ID studies from the starter set referenced; the cited papers ID is in n the second column referenced with the SLR ID and a capital C and a number; the title is in the third column, and the bibliographic reference is in the last column.

\begin{longtable}{e{}@{},{}@{}cp{9cm}p{3cm}@{}}
%% Cabeçalho da primeira página
\caption{Resultant Articles with design patterns methods}
\label{tab-snow}                                              \\[\belowcaptionskip]
\multicolumn{4}{@{}r@{}}{\textbf{(continue)}}                 \\[\belowcaptionskip]
\toprule%
\multicolumn{1}{@{}c}{\textbf{Key}}        &
\multicolumn{1}{c}{\textbf{Id}}            &
\multicolumn{1}{c}{\textbf{Title}}         &
\multicolumn{1}{c@{}}{\textbf{Ref}}        \\
\midrule%
\endfirsthead%
%% Cabeçalho das páginas (exceto primeira e última)
\caption[]{Resultant Articles with design patterns methods}   \\[\belowcaptionskip]
\multicolumn{4}{@{}r@{}}{\textbf{(continuation)}}             \\[\belowcaptionskip]
\toprule%
\multicolumn{1}{@{}c}{\textbf{Key}}        &
\multicolumn{1}{c}{\textbf{Id}}            &
\multicolumn{1}{c}{\textbf{Title}}         &
\multicolumn{1}{c@{}}{\textbf{Ref}}        \\
\midrule%
\endhead%
%% Cabeçalho da última página
\caption[]{Resultant Articles with design patterns methods}   \\[\belowcaptionskip]
\multicolumn{4}{@{}r@{}}{\textbf{(conclusion)}}               \\[\belowcaptionskip]
\toprule%
\multicolumn{1}{@{}c}{\textbf{Key}}        &
\multicolumn{1}{c}{\textbf{Id}}            &
\multicolumn{1}{c}{\textbf{Title}}         &
\multicolumn{1}{c@{}}{\textbf{Ref}}        \\
\midrule%
\endlasthead%
%% Rodapé da última página
\bottomrule%
\LTSourceOrNote{Own authorship (2023)}           \\
\endlastfoot% 

S1  &        &                                                                                                                                                                                                                                               &                                 \\
    & S1-C1   & Generation of refactoring algorithms by grammatical evolution                                                                                                                                                                                  & \citeauthor*{Mariani2022}     \\
    & S1-C2   & Refactorings and Technical Debt in Docker Projects: An Empirical Study                                                                                                                                                                         & \citeauthor*{Ksontini2021}    \\
    & S1-C3   & RefDetect: A Multi-Language Refactoring Detection Tool Based on String Alignment                                                                                                                                                               & \citeauthor*{Moghadam2021}    \\
    & S1-C4   & Supporting refactoring of BDD specifications—An empirical study                                                                                                                                                                                & \citeauthor*{Irshad2022}      \\
    & S1-C5   & Refactoring Techniques for Improving Software Quality: Practitioners’ Perspectives                                                                                                                                                             & \citeauthor*{Ksontini2021}    \\
    & S1-C6   & Automated refactoring of legacy JavaScript code to ES6 modules                                                                                                                                                                                 & \citeauthor*{Paltoglou2021}   \\
S2  &        &                                                                                                                                                                                                                                               &                                 \\
    & S2-C1   & Design and Implementation of a Web-Based Application for Code Smells Repository                                                                                                                                                                & \citeauthor*{Bamizadeh2021}   \\
    & S2-C2   & Object-Oriented Code Metric-Based Refactoring Opportunities Identification Approaches: Analysis                                                                                                                                                & \citeauthor*{Bassey2017}      \\
    & S2-C3   & Analysing The Effects Of Refactoring On Software Quality Attributes                                                                                                                                                                            & \citeauthor*{Singh2018a}       \\
    & S2-C4   & Measuring Code Smells and Anti-Patterns                                                                                                                                                                                                        & \citeauthor*{Reeshti2019}     \\
S5  &        &                                                                                                                                                                                                                                               &                                 \\
    & S5-C1   & Code Smell Refactoring for Energy Optimization of Android Apps                                                                                                                                                                                 & \citeauthor*{Reeshti2021}     \\
    & S5-C2   & Software Engineering Paradigm for Real-Time Accurate Decision Making for Code Smell Prioritization                                                                                                                                             & \citeauthor*{Singh2021}       \\
    & S5-C3   & A Framework to Improve Quality of a Java System by Performing Refactoring                                                                                                                                                                      & \citeauthor*{singhAndBindal2020}       \\
    & S5-C4   & Detecting Sudden Variations in Web Apps Code Smells’ Density: A Longitudinal Study                                                                                                                                                             & \citeauthor*{Rio2021}         \\
    & S5-C5   & Controlling software evolution process using code smell visualization                                                                                                                                                                          & \citeauthor*{Nabilah2019}     \\
    & S5-C6   & PHP code smells in web apps: survival and anomalies                                                                                                                                                                                            & \citeauthor*{Rio2021}         \\
    & S5-C7   & Bad Smell Detection Using Machine Learning Techniques: A Systematic Literature Review                                                                                                                                                          & \citeauthor*{Al-Shaaby2020}   \\
    & S5-C8   & Recovering Android Bad Smells from Android Applications                                                                                                                                                                                        & \citeauthor*{Rasool2020}      \\
    & S5-C9   & To improve code structure by identifying move method opportunities using frequent usage patterns in source-code                                                                                                                                & \citeauthor*{Singh2019}       \\
    & S5-C10  & Using software metrics to detect temporary field code smell                                                                                                                                                                                    & \citeauthor*{Gupta2020a}      \\
    & S5-C11  & Rank-based univariate feature selection methods on machine learning classifiers for code smell detection                                                                                                                                       & \citeauthor*{Jain2022}        \\
    & S5-C12  & TFfinder: A Software tool to discover Temporary Field code smell                                                                                                                                                                               & \citeauthor*{Gupta2020b}      \\
    & S5-C13  & Analysis of code smell to quantify the refactoring                                                                                                                                                                                             & \citeauthor*{Sehgal2017}      \\
S6  &        &                                                                                                                                                                                                                                               &                                 \\
    & S6-C1   & Does Code Complexity Affect the Quality of Real-Time Projects?: Detection of Code Smell on Software Projects using Machine Learning Algorithms                                                                                                 & \citeauthor*{Patnaik2021b}    \\
    & S6-C2   & A hybrid approach to identify code smell using machine learning algorithms                                                                                                                                                                     & \citeauthor*{Patnaik2021a}     \\
    & S6-C3   & Illustration and detection of exception handling bad smells                                                                                                                                                                                    & \citeauthor*{Tarwani2021}     \\
    & S6-C4   & Automated refactoring of legacy JavaScript code to ES6 modules                                                                                                                                                                                 & \citeauthor*{Paltoglou2021}   \\
S8  &        &                                                                                                                                                                                                                                               &                                 \\
    & S8-C1   & Enabling Decision and Objective Space Exploration for Interactive Multi-Objective Refactoring                                                                                                                                                  & \citeauthor*{Rebai2020}       \\
    & S8-C2   & Explainable Search-Based Refactoring                                                                                                                                                                                                           & \citeauthor*{Abid2021c}       \\
    & S8-C3   & Intelligent Change Operators for Multi-Objective Refactoring                                                                                                                                                                                   & \citeauthor*{Abid2021a}       \\
    & S8-C4   & Interactive Decision and Objective Space Exploration for Search Based Refactoring                                                                                                                                                              & \citeauthor*{Rebai2019}       \\
    & S8-C5   & A Many-Objective Estimation Distributed Algorithm Applied to Search Based Software Refactoring                                                                                                                                                 & \citeauthor*{Botelho2018}     \\
    & S8-C6   & X-SBR: On the Use of the History of Refactorings for Explainable Search-Based Refactoring and Intelligent Change Operators                                                                                                                     & \citeauthor*{Abid2021b}       \\
    & S8-C7   & The Effectiveness of Supervised Machine Learning Algorithms in Predicting Software Refactoring                                                                                                                                                 & \citeauthor*{Aniche2022}      \\
    & S8-C8   & Improving Readability of Scratch Programs with Search-based Refactoring                                                                                                                                                                        & \citeauthor*{Adler2021}       \\
    & S8-C9   & Untangling the Knot: Enabling Architecture Evolution with Search-Based Refactoring                                                                                                                                                             & \citeauthor*{Ivers2022}       \\
    & S8-C10  & Harnessing deep learning algorithms to predict software refactoring                                                                                                                                                                            & \citeauthor*{Alenezi2020}     \\
    & S8-C11  & DEPICTER: A Design-Principle Guided and Heuristic-Rule Constrained Software Refactoring Approach                                                                                                                                               & \citeauthor*{Zhao2022}        \\
    & S8-C12  & EASIER: An Evolutionary Approach for Multi-objective Software ArchItecturE Refactoring                                                                                                                                                         & \citeauthor*{Arcelli2018}     \\
    & S8-C13  & Unsupervised Learning For Refactoring Pattern Detection                                                                                                                                                                                        & \citeauthor*{Farah2021}       \\
    & S8-C14  & Model refactoring by example: A multi-objective search based software engineering approach                                                                                                                                                     & \citeauthor*{Ghannem2018}     \\
    & S8-C15  & A survey of many-objective optimisation in search-based software engineering                                                                                                                                                                   & \citeauthor*{Ramirez2019}     \\
    & S8-C16  & Applying design patterns in the search-based optimization of software product line architectures                                                                                                                                               & \citeauthor*{Guizzo2019}      \\
S9  &        &                                                                                                                                                                                                                                               &                                 \\
    & S9-C1   & An automated extract method refactoring approach to correct the long method code smell                                                                                                                                                         & \citeauthor*{Shahidi2022}     \\
    & S9-C2   & Automated Refactoring of Unbounded Queries in Software Automation Platforms                                                                                                                                                                    & \citeauthor*{Fernandes2021}   \\
    & S9-C3   & Cross-Project Software Refactoring Prediction Using Optimized Deep Learning Neural Network With the Aid of Attribute Selection                                                                                                                 & \citeauthor*{Panighrahi2022}  \\
    & S9-C4   & An automatic refactoring framework for replacing test-production inheritance by mocking mechanism                                                                                                                                              & \citeauthor*{Wang2021}        \\
    & S9-C5   & Refactoring Legacy Software for Layer Separation                                                                                                                                                                                               & \citeauthor*{Khalilipour2021} \\
S10 &        &                                                                                                                                                                                                                                               &                                 \\
    & S10-C1  & Composite Refactoring: Representations, Characteristics and Effects on Software Projects                                                                                                                                                       & \citeauthor*{Bibiano2022}     \\
S11 &        &                                                                                                                                                                                                                                               &                                 \\
    & S11-C1  & RefDiff4Go: Detecting Refactorings in Go                                                                                                                                                                                                       & \citeauthor*{Brito2020}       \\
    & S11-C2  & Test smell detection tools: A systematic mapping study                                                                                                                                                                                         & \citeauthor*{Aljedaani2021}   \\
    & S11-C3  & An automated extract method refactoring approach to correct the long method code smell                                                                                                                                                         & \citeauthor*{Shahidi2022}     \\
    & S11-C4  & Automated refactoring of legacy JavaScript code to ES6 modules                                                                                                                                                                                 & \citeauthor*{Paltoglou2021}   \\
    & S11-C5  & Automatic detection of Long Method and God Class code smells through neural source code embeddings                                                                                                                                             & \citeauthor*{Kovačević2022}   \\
    & S11-C6  & Are Code Smell Co-occurrences Harmful to Internal Quality Attributes?: A Mixed-Method Study                                                                                                                                                    & \citeauthor*{Martins2020}     \\
    & S11-C7  & How do Code Smell Co-occurrences Removal Impact Internal Quality Attributes? A Developers' Perspective                                                                                                                                         & \citeauthor*{Martins2021}     \\
    & S11-C8  & "Project smells" -- Experiences in Analysing the Software Quality of ML Projects with mllint                                                                                                                                                   & \citeauthor*{van2022}         \\
    & S11-C9  & Toward the automatic classification of Self-Affirmed Refactoring                                                                                                                                                                               & \citeauthor*{AlOmar2021b}     \\
    & S11-C10 & A framework to improve quality of a Java system by performing refactoring Currently I am working on Disaster Management and energy efficient deployment View project A framework to improve quality of a Java system by performing refactoring & \citeauthor*{Singh2020}       \\
    & S11-C11 & TERTIARY STUDY on LANDSCAPING the REVIEW in CODE SMELLS                                                                                                                                                                                        & \citeauthor*{Yaqoob2021}      \\
    & S11-C12 & MARS: Detecting brain class/method code smell based on metric–attention mechanism and residual network                                                                                                                                         & \citeauthor*{Zhang2021a}       \\
    & S11-C13 & RAID: Tool Support for Refactoring-Aware Code Reviews                                                                                                                                                                                          & \citeauthor*{Brito2021}       \\
    & S11-C14 & Code smells detection and visualization: A systematic literature review                                                                                                                                                                        & \citeauthor*{Pereira2022}     \\
S13 &        &                                                                                                                                                                                                                                               &                                 \\
    & S13-C1  & Consistency validation method for Java fine-grained lock refactoring                                                                                                                                                                           & \citeauthor*{Zhang2021b}         
\end{longtable}
\FloatBarrier

It was possible to analyze that some papers cited more than one SLR with slightly different subjects but the same refactoring matter. The article by \textcite{Paltoglou2021} focuses on proposing refactoring legacy Javascript to a newer version called ES6 with many more features and reliability. The study appears on S1, S6, and S11.


\section{Trends}
\label{sec-trends}
This study used forward snowballing to propose a set of papers to be candidates for its original SRLs and SMRs.
A total of 65 papers could be included in their respective SRLs as the final step of the snowballing, considering the acceptance criteria of the original papers, which strictly ensures quality. In conclusion, snowballing showed an excellent and fast way to update an SRL and SMR.
The paper also shows how SLRs can age pretty well, like S9, a work from 2020 that already has 40 citations, where, after filtering, five possible studies were achieved to include in an update of this paper.

As argued, SRLs age quite rapidly; the same will happen with work as new papers are published every month, and some of those are likely to cite an article from the starter set, assuming that to avoid paper aging, they ought to apply an update of the database search and snowballing in a fixed data frame.

\section{Closing Remarks}
\label{sec-cloasing-remarks}

The main application of snowballing in this work is to find new methods to find opportunities to refactor Java code to apply design patterns by finding all refactoring-related SLRs and SMRs and applying the snowballing techniques. After the process, no new article was found on the desired topic. That shows the importance of RMT and how a tool that can concentrate all refactoring methods to design patterns has not yet been found in the academy and can change how developers apply refactorings to their projects.

%% Capítulo 4
\chapter{Methodology used to refactor and improve RMT tool}%
\label{methodology}

This chapter explores the methodology for refactoring and improving the RMT tool. It delves into various techniques and best practices for improving and optimizing the tool's performance and maintainability. The discussion covers a range of refactoring strategies.

\textcolor{red}{creio que aqui deveria ter uma figura com o overview da metodologia}

\textcolor{red}{faltou comentar sobre cada capítulo como fez para os outros}

As all services have been renamed in \cref{tab-services-map}, the corresponding nomenclature equivalences are described. The explanations for the renaming are provided in the subsequent chapters.

\begin{table}[h]
    \centering
    \caption{Original and New Names}
\begin{tabular}{c c}
        \toprule
        \textbf{Original Name} & \textbf{New Name} \\
        \midrule
        Detection Service & Detection And Refactoring Service \\
        Metrics Service & Metrics Calculation Service \\
        Intermediary Service & Projects Sync BFF \\
        \bottomrule
    \end{tabular}
    \label{tab-services-map}
    \SourceOrNote{Own authorship (2024)}
\end{table}

\section{Architectural Analysis of Packages from the RMT}
\label{sec-archtectural-analysis}

To understand the operational dynamics of RMT, it was imperative to analyze the codebase, specifically the various packages. The initial phase of the updating process involves discerning the functionality of different code segments to determine whether a code refactoring or a comprehensive service rewrite is necessary.

The analysis was carried out on the packages Intermediary Service, Detection Service, and Metrics Service, described below.

\subsection{Intermediary Sevice Packages Analisys}
\label{sub-intermediary-packages}
The intermediary service is systematically divided into four main packages, each comprising functionalities. The first functionality involves the management of refactoring projects. The second pertains to the facilitation of inter-service communication. The third serves as a service discovery mechanism by registering the addresses of all ancillary services to enable seamless subsequent communications. They are illustrated by the package diagram shown in \Cref{fig-package-intermediary}.

\begin{figure}[ht!]
\SetCaptionWidth{\textwidth}
\caption{Intermediary Service Package Diagram}
\label{fig-package-intermediary}
\includesvg{Chapter-4/Figures/intermediary-service.svg}
\SourceOrNote{Own authorship (2024)}
\end{figure}
\FloatBarrier

The package \texttt{datastore} includes configuration files relevant to the database pool and the connection configuration.

Package \texttt{files} contain repository files for database interactions, facilitating queries, insertions, and additional data manipulations.

Within the manager package, the entire outbound logic refers to the refactoring process and the discovery of services, encompassing all related requests for refactoring and generating metrics in the package \texttt{managers.projects} and registering services in the package \texttt{managers.members}.

For handling communication, the \texttt{ws.boundaries} package includes the controller configurations. Within this package, business logic is assigned to the persistence and querying of projects and dispatching requests to other services. Consequently, this package must access the \texttt{managers} and \texttt{files} packages.

\subsection{Detection Service Package Analisys}
\label{sub-detection-packages}

The detection service is arranged into six main packages, the core functionalities of which are periodically communicated with the intermediary service to ensure registration with the service discovery mechanism. In addition, interfaces have been developed to analyze the source code for potential refactoring candidates, and interfaces have been designed to perform refactoring on projects that contain such candidates. The package diagram displayed in \Cref{fig-package-detection} illustrates them.
\begin{figure}[ht!]
\SetCaptionWidth{\textwidth}
\caption{Detection Service Package Diagram}
\label{fig-package-detection}
\fontsize{7.5}{9.5}\selectfont
\includesvg[width =\textwidth]{Chapter-4/Figures/detection-service.svg}
\SourceOrNote{Own authorship (2024)}
\end{figure}
\FloatBarrier

The package \texttt{datastore} is configured with identical database parameters to those defined in \cref{sub-intermediary-packages}.

Package \texttt{ repository.project} has logic to manipulate the database where projects are saved and retrieved for unrefactored and refactored projects. The database configuration, such as the address and ports, is imported from the \texttt{datastore} package.

The package \texttt{methos.dataExtractoins} includes preconfigured interfaces for implementing various code extraction methods. The Abstract Syntax Tree is implemented as the extraction method for the refactoring processes within the RMT. Following the code transformation into an Abstract Syntax Tree (AST), the service accesses the files within the \texttt{domain.mehtos} to perform its designated function.

The package \texttt{managers.pulse} includes the configuration for the service registry, sending requests every minute to ensure its functionality \textcolor{red}{<<realmente é proof of life?>>} to the intermediary service, and can receive requests. Information, such as the address and port sent from the service, is retrieved from the package \texttt{domain.identity}.

The interfaces for candidates searching and refactoring projects are in the \texttt{domain.methdos} has interfaces to implement and extend the tool refactoring options. As the current methods implement the Abstract Syntax Tree as an extraction method, the \texttt{doamin.dataExtraction.utils} package has methods to facilitate AST manipulation.

To start refactoring, the controllers must receive an HTTP request on the \texttt{ws.boundaries} that distributes the request based on its URL path among the other class functions. 

\subsection{Metrics Service Package Analisys}
\label{sub-metrics-packages}

The metrics service is divided into six main packages and two main functionalities; as the detection service, the metrics service communicates with the intermediary service for the service registry; the second functionality calculates metrics and quality attributes (calculated with the metrics results). The services have interfaces to increase the available metrics and quality attributes. The \cref{fig-package-metrics} shows the diaram package.

\begin{figure}[ht!]
\SetCaptionWidth{\textwidth}
\caption{Metrics Service Package Diagram}
\label{fig-package-metrics}
\fontsize{9}{10}\selectfont
\includesvg[width =\textwidth]{Chapter-4/Figures/metrics-service.svg}
\SourceOrNote{Own authorship (2024)}
\end{figure}
\FloatBarrier

Consistent with the two preceding services, the \texttt{ datastore} is the repository for all database configuration settings.

To access the unrefactored and refactored projects, the package \texttt{repository.project} has all the logic queries to the database using the information provided by the \texttt{datastore} package.

Correspondingly to the previous service, the package \texttt{managers.pulse} encompasses the comprehensive logic required for registration within the service discovery mechanism. This service adheres stringently to all the specifications delineated in the detection service.

Similarly to the detection service, the \texttt{processor.qualityAttributes} has the interfaces to implement different methods of measuring code metrics and quality attributes. The interfaces' logic and calculations are in the \texttt{domain.metrics} and \texttt{domain.qualityAttribute} packages.

The classes with logic and calculations for generating metrics are in the \texttt{domain.metrics} package; for now, they are hardcoded, implementing the CK module created by \textcite{ck}; however, the interface is designed to integrate additional metrics generation methodologies in the \texttt{domain.qualityAttributes} package is located in the calculations for quality attributes, such as maintainability, readability, etc.

Consistent with previous services, the \texttt{ws.boundaries} packages serve as the access point for service functionalities. They coordinate computations by interfacing with the database to retrieve project data and invoke methods within the \texttt{processor.qualityAttributes} packages, thereby generating metrics and quality attributes.
 
\section{Creating Unit Tests}
\label{sec-improving-tool}

Upon comprehending the tool's functionalities, the subsequent step involves verifying the presence of tests. This is essential to commence the refactoring process, as explained by Fowler:

\Citation[english]{\cite[9]{fowler2018refactoring}}{Whenever I do refactoring, the first step is always the same. I need to ensure I have a solid set of tests for that section of code. The tests are essential because even though I will follow refactorings structured to avoid most of the opportunities for introducing bugs, I’m still human and still make mistakes}.

The RMT lacked any form of testing; therefore, according to \textcite{fowler2018refactoring} philosophy, the tool was subjected to unit tests to verify consistent behavior after refactoring. 

The tests focused on the business logic of the detection service and the quality attribute computations within the metrics service. Testing was bypassed for the intermediary service due to its redevelopment.

The tests were designed exclusively for classes with specific logic, excluding interfaces, enumerations (enums), and Plain Old Java Objects (POJOs). This exclusion is justified, as such classes only exhibit the intrinsic behavior provided by the programming language itself without incorporating any additional logic. Consequently, there is no need to test these classes.

\subsection{Designing Test Scenarios for Detection Service}
\label{detection-design-tests}

Tracing the tool's execution path, the initial classes slated for testing are within the \texttt{methods} package, as they are the classes responsible for parsing the code into an Abstract Syntax Tree (AST). 

After refactoring, the parsing behavior must remain consistent, given that the AST is the foundational structure enabling the tool's capacity to manipulate Java classes to refactor. The \cref{fig-class-detection-methods} illustrates the simplified class diagram for the Methods package.

\begin{figure}[ht!]
\SetCaptionWidth{\textwidth}
\caption{Detection Service Methods Package Simplified Class Diagram}
\label{fig-class-detection-methods}
\fontsize{7}{8}\selectfont
\includesvg[width =\textwidth, scale=1.0]{Chapter-4/Figures/detection-service-methods.svg}
\SourceOrNote{Own authorship (2024)}
\end{figure}
\FloatBarrier

The initial phase of the RMT refactoring process, 'data extraction,' involves parsing Java files into an Abstract Syntax Tree (AST), a task performed by the \texttt{AbstractSyntaxTree} class. The class testing methodology consists of the accuracy of the parsing process, the generation of an AST object, or the identification of errors as the library \cite{javaparser} executes the process. Therefore, the reliability of this library is assumed, necessitating only the validation of the output. This is achieved by converting a code sample into a string and confirming the equivalence between the AST output, also as a string, and the original input. Error conditions are tested by deliberately invoking the library with incorrect Java code and asserting the results.

The \texttt{AbstractSyntaxTreeFork} class orchestrates the refactoring implemented methods that use the Abstract Syntax Tree (AST) as the parsing mechanism. The refactoring process is divided into two primary stages: invoking methods to identify candidate elements and executing methods to refactor the identified candidate classes. Additionally, the class has logic for database manipulation; this aspect was not subjected to testing due to the substitution of all database communication mechanisms and the database itself, thereby preventing the necessity to preserve any pre-existing behavior. For testing the class, the refactoring techniques had to be mocked (creating an object that simulates the original object's behavior) and assuring the behavior when the methods return an error or success.

The \texttt{DetectionMethodsManagerImpl} was excluded from the testing due to the planned complete reimplementation of the class, which encapsulates the algorithms for project retrieval from the database and the execution of procedures within \texttt{AbstractSyntaxTreeFork}.

The methods executed by \texttt{AbstractSyntaxTree} are situated within the \texttt{methods} package and hold the logic for each implemented refactoring method. The classes are split into two distinct categories: the first category contains classes designed to identify refactoring candidates by detecting specific patterns within Java code that qualify for refactoring; the second category consists of classes intended to execute the refactoring process, thereby effectuating the requisite modifications to the code. The simplified class diagram is divided into two \cref{fig-class-detection-domain-wei} and \cref{fig-class-detection-domain-zafeiris} representing these classes.

\begin{figure}[ht!]
\SetCaptionWidth{\textwidth}
\caption{Detection Service Domain Package Simplified Class Diagram For Wei Related Files}
\label{fig-class-detection-domain-wei}
\fontsize{4}{5}\selectfont
\includesvg[width =\textwidth]{Chapter-4/Figures/detection-service-domain-wei.svg}
\SourceOrNote{Own authorship (2024)}
\end{figure}
\FloatBarrier

The \cref{fig-class-detection-domain-wei} illustrates the class diagram for the \cite{liu2014automated} method, whereas the \cref{fig-class-detection-domain-zafeiris} outlines the class diagram for \cite{zafeiris2017automated} method.

\begin{figure}[ht!]
\SetCaptionWidth{\textwidth}
\caption{Detection Service Domain Package Simplified Class Diagram For Zafeiris Related Files}
\label{fig-class-detection-domain-zafeiris}
\fontsize{5}{8}\selectfont
\includesvg[width =\textwidth]{Chapter-4/Figures/detection-service-domain-zafeiris.svg}
\SourceOrNote{Own authorship (2024)}
\end{figure}
\FloatBarrier

The classes \texttt{WeiEtAl2014Candidate}, \texttt{WeiEtAl2014FactoryCandidate}, \texttt{ZafeirisEtAl2016Candidate}, and \texttt{WeiEtAl2014StrategyCandidate} were excluded from testing, as previously mentioned since Plain Old Java Objects (POJOs) devoid of any business logic were not subjected to testing. The \texttt{WeiEtAl2014} and \texttt{ZafeirisEtAl2016} classes were also excluded from testing as they had been rewritten.

The initial tested class was the \texttt{AstHandler}, serving as an encapsulating wrapper that facilitates direct access to an AST branch. The test cases must ensure that the methods can access the AST objects, such as methods, variables, inner classes, etc. The class also has some methods with limited logic, like assuring that two variables are the same and finding a class's parent, among others. Those functionalities were tested to ensure that they worked as intended and would continue after the refactoring. Some opportunities for improvement were discovered during the development of test cases, which are addressed in \cref{results}. 

The tests for the following classes are categorized into two distinct groups: verifiers and preconditions and executors. The preliminary classes subjected to testing were related to the \textcite{liu2014automated} methods, specifically \texttt{WeiEtAl2014FactoryVerifier}, \texttt{WeiEtAl2014StrategyVerifier}, and 	\texttt{WeiEtAl2014StrategyVerifier} and \texttt{LiteralValueExtractor}. It is intrinsic to the verifier to have methods that return a boolean, so the test must ensure that the method returns true if the case has the right condition and returns false if it is wrong. Under the analogous implementation approach brought about by the \textcite{zafeiris2017automated} method, the subsequent classes, namely, \texttt{ZafeirisEtAl2016Verifier}, \texttt{ExtractMethodPreconditions}, \texttt{SiblingPreconditions}, and \texttt{SuperInvocationPreconditions}, were subjected to identical rigorous tests.

The following tests targeted executors in both refactoring methods, beginning again with the paper by \textcite{liu2014automated} with specific classes tested included \texttt{WeiEtAl2014FactoryExecutor} and \texttt{WeiEtAl2014StrategyExecutor}; concerning the 	\textcite{zafeiris2017automated} article, the scrutinized class was \texttt{ZafeirisEtAl2016Executor}. The testing methodology for these executors is straightforward; within each referenced publication, the authors illustrate the functionality of the refactoring approaches by providing authentic code exemplars. These examples were used to verify that the output generated by the executors corresponded precisely to the outputs delineated in the respective papers.

\subsection{Designing Test Scenarios for Metrics Service}
\label{metrics-design-tests}

Most classes within the Metrics services require extensive rewrites to integrate the new architectural framework. Consequently, testing was limited to only three classes represented in \cref{fig-class-metrics-quality}, excluding \texttt{QualityAttributeMetric} because it is a POJO.

\begin{figure}[ht!]
\SetCaptionWidth{\textwidth}
\caption{Metrics Service Simplified Class Diagram for Tested Classes}
\label{fig-class-metrics-quality}
\includesvg{Chapter-4/Figures/metrics-service-quality-attributes.svg}
\SourceOrNote{Own authorship (2024)}
\end{figure}
\FloatBarrier

The critical aspect of testing metrics and quality attributes for refactoring lies in maintaining consistent calculations for metrics and quality attributes. This ensures that the test's integrity remains intact, notwithstanding any alterations in the library that may modify the metric computation method.

Performing an in-depth analysis for each class. The \texttt{Metric} class extracts metrics from the \textcite{ck} library, incorporating these metrics for each class within a project. The test must ensure the continuity of the metric calculations for each class using identical metrics derived from the CK library. The \texttt{Proportion} class encompasses two calculation methodologies: the \texttt{direct} method, entailing the division of the refactored value by the original, and the \texttt{inverted} method, which involves the division of the original value by the refactored; testing must ensure that these calculations yield consistent results after refactoring. The \texttt{QualityAttribute} class served as the foundational model to calculate the quality attributes used in RMT. This was achieved by integrating the results of specific metrics with proportion calculations; the testing process must confirm that identical metric values consistently yield the same quality attribute values, even after subsequent refactoring.

\section{Enhancing Software: Refactoring and Restructuring Techniques}

After creating tests, the tool can be refactored and restructured. The first step in refactoring the RMT was to update the Java version to 21 since the current version was 8, released in 2014; the new version brings new features to the language and is faster \cite{java21}. 

Despite the advantages of upgrading the Java JDK, one of the fundamental tenets of the refactoring initiative was to transition the framework to Spring Boot. Consequently, since the latest version of Spring Boot is exclusively compatible with Java 17 and higher. All dependencies were also updated to the latest version.

IntelliJ, developed by JetBrains, was the integrated development environment (IDE) for refactoring the tool \cite{intellij}. IntelliJ has an "inspection" functionality that proactively recommends refactorings of the current code, with the primary objective of enhancing it \cite{intellij-inspection}. In the RMT code, suggestions were made to make the code more readable, such as changing an expression from a negated to a regular expression, as exemplified in \cref{alg-intellij}. The IntelliJ automation subjected the metrics and detection services to a comprehensive refactoring process.

\begin{algorithm}[!htbp]
\caption{Exemple of a code refactoring by IntelliJ}%
\label{alg-intellij}
\begin{algorithmic}[1]
\STATE{var a = Optional.ofNullable(variable)}
\STATE{if(!a.isPresent()) \{...\}} \COMMENT{Exemple of code IntelliJ highlights to refactor}
\STATE{if(a.isEmpty()) \{...\}} \COMMENT{Code after refactoring}
\end{algorithmic}
\SourceOrNote{Own Authorship (2024)}
\end{algorithm}
\FloatBarrier

The Spring Framework was integrated into the detection and metrics services to initiate the early refactoring. Concurrently, all dependencies associated with the legacy database, including files specific to its utilization, were removed. The prescribed method for framework application began with the controllers (even though they would be removed afterward) and subsequently extended to the imported classes. It was not requisite for all classes to undergo modifications to align with the refactoring; however, classes exhibiting ambiguous behavior were earmarked for subsequent review during the functionality improvement phase. The tool was no longer functional, but previously implemented tests backed its functionality.

The sequence for restructuring the tool begins with the complete rewrite of the intermediary service, as it is the only service that receives requests from the user following the new architecture. Refactoring progresses to the detection service as it gets the message sent by the intermediary service, processes it, and sends it to the metrics service if any candidate is found. The process is complete in the metrics service, which is the last service in the architecture.


%Iniciei tentando adiconar coisas de novas versões do Java pro codigo ficar mais limpo e atualizar a dependencias.
%Refatorei de acordo com as dicas do intellij.
%Adicionar o novo framework nos 2 serviços.
%Explicar por qual modulo comecei a refatorar e porque. Durante a refatoração do modulo confrome as funcionalidades foram implementadas para aperfeiçoa-las os pacotes e classes necessarios foram sendo criados.

%Pegar as limitacoes e explicar como elas foram refatoradas
\subsection{Restructure of Intermediary Service}
\label{restruct-intermediary}

As previously mentioned, the intermediary service underwent a comprehensive restructuring due to the obsolescence of most of its functionalities according to the new design. 

The remaining essential features were also rewritten but updated to integrate current technologies. The rewrite addresses some limitations of the RMT by focusing on keeping the service simple implementation and the asynchronous management of the tool. The intermediary service is assigned to receive a project via HTTP, save the project information in the event database, the project file on an object storage tool, send the project ID to the detection service and close the requests successfully. The tool also offers an endpoint to consult the refactoring status, as the process is asynchronous.

Those modifications attempt to solve the service freezing issue and remove many responsibilities from it. The freezing is avoided because the service is ready to receive a new request after sending a message to the detection service without waiting for the service process to respond. Concerning responsibilities, the service does not have a load balancer or a service registry functionality, which helps to save machine resources.

As the tool is no longer an intermediary service, the name was changed to Project Sync BFF (back-end for front-end). The name implies that the service offers APIs to sync the project and, therefore, an endpoint to register the project and others to verify the project status. The service also provides an endpoint that creates a zip file with the chosen refactorings and returns a link to download them.

%Mudei o nome e justicar. 
%Mudei a formal de funcionamento e porque

\subsection{Detection Service Refactoring}
\label{restruct-detection}

The detection service refactoring began with replacing and removing obsolete files, including substituting the controller with a queue consumer and replacing previously deleted database files with new ones, contemplating the new architecture technologies. Following the replacement of the controllers and the integration of the new database access files, the effort to address the service limitations began.

The main limitation of the service was the disk access problems, in which a refactoring would access the disk many times to save and parse the files again. To address the problem, a new object was created to store the files to be used solely in memory. While replacing old objects, some optimizations \textcolor{red}{Quais otimizações?} were done by removing duplicated codes and only parsing the Java files once for each refactoring method.

The name was changed from the detection service to the detection and refactoring service as it detects candidates and refactors.


%Erros econtrados nos testes

\subsection{Metrics Service Refactoring}
\label{restruct-metrics}

Refactoring started as in the last service, in the service metrics, by replacing the controller with a queue consumer and the database repository files with the new ones. As a known limitation, the CK library \textcite{ck} was removed from hard-coded as a Maven dependency to facilitate management and updates. 

The service also had some logic for the metrics and quality attributes implemented in Enums; this logic was reimplemented in a strategy pattern and improved by calculating the metrics only once for each candidate instead of three times as in the previous code.

Following the other two services, the name was also changed from metrics service to metrics calculation service as it is a more meaningful name. 

\textcolor{red}{para dar um nome mais significativo}

%Removi a CK que estava no codigo, atualizei e importer via package manager
%adicionei novos padrões

\section{Tests in the RMT 2.0 tool}
\label{test-rmt}

Despite the presence of unit tests following the refactoring process, the tool must undergo complete system-level tests. A Dockerfile was generated for each service to establish the integration test environment, which allows containerization. To facilitate the local execution of AWS services, Localstack enables SQS and S3 services, with Redis provisioned in a separate container. The deployment of all these containers is orchestrated using a docker-compose file.

Upon configuring the experimental environment, the tests were executed using the sample files provided by each respective article of the methods implemented in RMT 2.0. 

During the initial testing phase, the tool failed to refactor all files within the project. The issue was traced to the \texttt{getIfStatements} method within the AstHandle class. This method erroneously returned a single if statement rather than a comprehensive list of if statements, resulting in the refactoring being applied exclusively to the first if statement in the \cite{liu2014automated} method. Upon fixing the method, subsequent validation tests confirmed the tool's functionality as expected.

\textcolor{red}{acho que não ficou muito claro, creio que deva realizar os testes e depois voltar aqui para escrever}

\section{Cloasing Remarks}


%Escrever como foi testada a ferramenta após ser refatorada

%% Parte 3 (elemento opcional; grupo de capítulos)
% \part{Conclusão}%
% \label{part:concl}

%% Capítulo 5
\chapter{The RMT 2.0}
\label{results}
\textcolor{red}{1 - RMT 2.0\\
2 - Testes com projetos\\
3 - Comparação da RMT 1 com 2
}

\section{The Renewed Cloud Based Architecture}
\label{sec-cloud}
The revised architecture introduces several modifications to the communication protocols between services. The initial version of the tool had three services and one Java desktop application. The revised architecture transitioned the desktop application to a web-based platform, altering each service's operational dynamics. The newly devised architecture is illustrated in \Cref{fig-architecture}.

\begin{figure}[ht!]
\SetCaptionWidth{\textwidth}
\caption{RMT revised architecture diagram}
\label{fig-async}
\includegraphics[width =\textwidth, scale=0.2]{Chapter-5/Figures/Async.png}
\SourceOrNote{Own authorship (2024)}
\end{figure}
\FloatBarrier

\subsection{Comunication Improvements}

The communication type was switched to queues instead of HTTP requests. The modification primarily aims to establish a communication mechanism with enhanced configurability in the event of failures, as the services must acknowledge the messages, or they will be retried as often as configured. Queues also offer an asynchronous communication that is out of the box. Otherwise, in the HTTP request, the retry has to be implemented on the client side according to the server response. It is also possible to have an HTTP asynchronous request, but it must be implemented on the client side.

The Simple Queue Service (SQS) operates in its default configuration, implementing a quintuple retry mechanism to transmit a JSON object containing the project identifier between microservices.

\subsection{Storage \& Database Improvements}

The initial version of RMT utilizes MongoDB GridFs for file storage, a feature designed to facilitate the preservation of files within the database architecture. The updated version uses Amazon S3 for project storage, using a dual-bucket strategy: one designated for the original project and the other for the refactored iterations. All microservices are granted access to the designated buckets, enabling seamless file storage and retrieval. The project information and statuses are stored in the Redis in-memory document database, and all services have access.

\section{RMT 2.0 Package Arrangement}

\begin{figure}[ht!]
\SetCaptionWidth{\textwidth}
\caption{Project Sync BFF package diagram}
\label{fig-project-sync-package}
\fontsize{8}{10}\selectfont
\includesvg[width =\textwidth]{Chapter-5/Figures/project-sync-bff.svg}
\SourceOrNote{Own authorship (2024)}
\end{figure}
\FloatBarrier


\begin{figure}[ht!]
\SetCaptionWidth{\textwidth}
\caption{Detection and Refactoring service package diagram}
\label{fig-detection-refactoring-package}
\includesvg[width =\textwidth]{Chapter-5/Figures/detection-and-refactoring.svg}
\SourceOrNote{Own authorship (2024)}
\end{figure}
\FloatBarrier

\begin{figure}[ht!]
\SetCaptionWidth{\textwidth}
\caption{Metrics Calculator service package diagram}
\label{fig-metrics-calculator-package}
\includesvg[width =\textwidth]{Chapter-5/Figures/metrics-calculator.svg}
\SourceOrNote{Own authorship (2024)}
\end{figure}
\FloatBarrier



\section{Services Behavior Improviments}
\label{sub-services-behaivour}

Notwithstanding recent updates to the tool, its resultant behavior remains unchanged. The tool now runs with a linear execution flow, obviating the need for interservice communication during project refactoring. Given the process's asynchronous nature, users must request the completion status to display the retrieved relevant information. The process is represented in \cref{fig-activity-diagram}, with the blue nodes representing the behavior inherited from the initial version of the tool.

\begin{figure}[ht!]
\SetCaptionWidth{\textwidth}
\caption{RMT 2.0 Activity Diagram}
\label{fig-activity-diagram}
\fontsize{5.8}{8}\selectfont
\includesvg[width =\textwidth]{Chapter-5/Figures/activity-refactored-rmt.svg}
\SourceOrNote{Own authorship (2024)}
\end{figure}
\FloatBarrier

The flow starts with the user selecting the project and sending it to the API Gateway connected to the Project Sync BFF; the service sends the project ID to the refactoring queue to start the process. The Detection and Refacred Service will immediately search for the candidates to be refactored and apply the refactoring. A message with the project ID is sent to the metrics calculation queue if any candidate is found. The Metrics Calculation Service computes the quality attributes for the project and ends the processing. The Project Sync BFF service offers a pull mechanism that consults Redis looking for a final status and, if found, returns the project information and metrics. The behavior is displayed in \Cref{fig-activity-diagram}.

\section{Testing RMT 2.0}
\label{sub-tests}

\subsection{Testing Through Refactored Article Application Exemplars}

The implementations exemplified by \textcite{liu2014automated} and \textcite{zafeiris2017automated} were used for the preliminary evaluation.

To refactor using the factory method design pattern, \textcite{liu2014automated} introduces four distinct classes: a \texttt{Logger} interface, 	\texttt{FileLogger} and \texttt{DatabaseLogger} which implement the interface, and the \texttt{LoggerFactory} serving as a factory to select among the implementations. The \cref{fig-factory-client} displays the refactored \texttt{LoggerFactory} that was changed to abstract; the implementation was moved to the \texttt{FileLoggerFactory} and \texttt{DatabaseLoggerFactory}. The metrics displayed are positive, indicating an improvement in maintainability and reusability without changing reliability.

\begin{figure}[ht!]
\SetCaptionWidth{\textwidth}
\caption{Refactored Project For Factory Method}
\label{fig-factory-client}
\includegraphics[width =\textwidth]{Chapter-5/Figures/rmt-factory-client.jpeg}
\SourceOrNote{Own authorship (2024)}
\end{figure}
\FloatBarrier

To exemplify the strategy pattern \textcite{liu2014automated}, create the \texttt{MovieTicket} class, which has an if statement for each ticket type. The method creates an abstract \texttt{Strategy} class with the calculate method, the code extracted from the \texttt{MovieTicket} is implemented on the \texttt{ConcreteStrategyS}, \texttt{ConcreateStrategyM} and \texttt{ConcreateStrategyC}. The maintenance and reusability metrics improved, although reliability decreased, as shown in \cref{fig-strategy-client}.

\begin{figure}[ht!]
\label{fig-strategy-client}
\caption{Refactored Project For Strategy}
\includegraphics[width =\textwidth]{Chapter-5/Figures/rmt-strategy-client.png}
\SourceOrNote{Own authorship (2024)}
\end{figure}
\FloatBarrier

The \textcite{zafeiris2017automated} implements the template method using the jade-test-suit to exemplify the process. The \texttt{JICPPeer} class is the parent of the \texttt{JICPSPeer} class, which implements a method that has a \texttt{super} call; the refactoring extract code before and after the super call into new methods and replaces the super call to a method call with the same behavior. The maintainability, reusability, and reliability metrics are negative, showing a deterioration. It is represented in \cref{fig-template-client}

\begin{figure}[ht!]
\label{fig-template-client}
\caption{Refactored Project For Template Method}
\includegraphics[width =\textwidth]{Chapter-5/Figures/rmt-template-client.png}
\SourceOrNote{Own authorship (2024)}
\end{figure}
\FloatBarrier

An additional approach to test the tool and ensure its quality will undergo a self-scanning procedure. The analysis was performed to test the effectiveness of the tool, with a focus on evaluating RMT and its next version, RMT 2.0. Examining both RMT versions showed that there were no elements that needed refactoring. 

\section{Comparing RMT versions}

\section{Closing Remarks}
\label{sec-closingproposal}


%% Capítulo 6
\chapter{Conclusion}%
\label{conclusion}

RMT 2.0 guarantees superior performance, scalability, and reliability through its modern cloud-based architecture. This evolution accommodates more extensive and intricate projects, ensuring a more efficient and resilient development environment. The transition from conventional HTTP requests to asynchronous communication via message queues refines interactions, ensuring more robust communication. Furthermore, storage migration to Amazon S3 and Redis provides a more scalable and reliable solution, facilitating refactored project distributions.

Using the Spring framework, the restructured package architecture simplifies the codebase, making it more manageable and maintainable. This architectural refinement reduces complexity and speeds up development, allowing developers to focus on coding rather than intricate configurations. A new browser-based interface enhances the user experience, making project evaluation and refactoring easier to access.

The initial testing of RMT 2.0 addressed and solved issues along the services. In contrast, further testing with redesigned and real-world projects confirmed the tool's effectiveness in identifying refactoring candidates and measuring the resultant with software metrics. Comparative analyses demonstrated that RMT 2.0 significantly outperforms the first version. These improvements resulted in a notable 63.64\% increase in overall execution time, making the tool more practical and efficient for developers.

Despite these advancements, the deployment of RMT 2.0 has been greatly simplified with Docker containerization, accompanied by automated scripts that simplify the setup process. These enhancements to the tool translate into substantial benefits for developers, including easier deployment, improved performance, and greater accessibility.

\section{Future Work}

As the approach created has various extension points, the following suggestions for future work are proposed:
\begin{itemize}
  \item Enhance the Detection and Refactoring Service by integrating additional refactoring methodologies that emphasize the implementation of design patterns; 
    \item Conduct comprehensive evaluations for new quality attributes.
\end{itemize}

Despite the advancements rendered within the tools, certain constraints remain evident:
\begin{itemize}
  \item Further optimization of the Template Method refactoring to enhance computational efficiency. 
  \item Segregation of method detection and refactoring processes into individual threads to enable parallel execution.
  \item Develop an enhanced visualization methodology for refactoring candidates within the same class and an improved mechanism for integrating various refactorings in one class to mitigate potential conflicts.
\end{itemize}

%% Marcadores de PDF para capítulos subsequentes no nível principal
% \PhantomPart%

%% Formatação de elementos pós-textuais (backmatter)
%% Comando \PostTextual*: remove as seções de nível inferior às primárias do
%% sumário e dos marcadores de PDF.
\PostTextual%% Não comentar ou remover

%% Referências
\PrintReferences%

%% Glossário (elemento opcional)
%%%% Opção 1: makeindex; conforme o [Arquivo de Entradas] em \MakeGlossary.
%%%% Comando \PrintGlossary*: remove o indicativo (páginas) dos itens.
% \PrintGlossary%[\bfseries]%% Estilo de fonte do termo (opcional)
%%%% Opção 2: manual; editar o {Arquivo} para alterar.
% \input{./Post-Textual/Optionals/glossary-list}

% Parte Apêndices (elemento opcional; grupo de capítulos)
% \AppendicesPart%

%% Apêndices (elemento opcional)
\begin{Appendices}
\include{./Post-Textual/appendix-a}
\include{./Post-Textual/appendix-b}
\end{Appendices}

% Parte Anexos (elemento opcional; grupo de capítulos)
% \AnnexesPart%

%% Anexos (elemento opcional)
\begin{Annexes}
\include{./Post-Textual/annex-a}
\include{./Post-Textual/annex-b}
\end{Annexes}

%% Índice remissivo (elemento opcional)
%%%% Opção 1: makeindex (\index{...}, \Index{...}, etc.)
% \PrintIndex%
%%%% Opção 2: manual; editar o {Arquivo} para alterar.
% \input{./Post-Textual/Optionals/index-list}

%% Fim do documento
\end{document}%% Não comentar ou remover

%% Observação: o arquivo final (PDF) pode ser convertido para PDF/A usando
%% diversas ferramentas, por exemplo:
%%   https://www.pdfforge.org/online/en/pdf-to-pdfa
