%%%% ELEMENTOS PRÉ-TEXTUAIS
%%
%% Parte que antecede o texto com informações que ajudam na identificação e na
%% utilização do trabalho.
%%
%% Observações:
%% 1. {Arg} argumento obrigatório de ambiente ou comando.
%% 2. [Arg] argumento opcional de ambiente ou comando.

%% Folha de rosto
%% Contém os elementos essenciais à identificação do trabalho, além de uma
%% licença Creative Commons (https://creativecommons.org/choose/).
%% Ambiente {TitlePage*}: aplica caixa alta no título em idioma secundário.
\begin{TitlePage}%% Argumentos (2):
[BY]%% Tipo de licença (BY, BY-SA, BY-ND, BY-NC, BY-NC-SA ou BY-NC-ND)
% [Texto da licença]%% Substitui o texto padrão para cada tipo de licença
%%%% Descrição do trabalho (padrão; alterar se necessário)
\DocumentTypeName\ presented as requirement to obtain the title of \StudentsTitlesList\ in \CourseName\ from \ifbool{MakeAcr}{\intldescr{UTFPR} (\intl{UTFPR})}{\UTFPRName\ (UTFPR)}.
%%%% Orientador(es) (de 1 a 3): {Número}; {Dados}
\Advisor{1}{%
   Gender   = {Female},%% Ou {Female}
   Title    = {\ProfCall\ \PhDCall},%% {\<T>Call}; <T>: Prof/PhD/DSc/MSc/Eng
   Fullname = {Simone Nasser Matos},%% Conforme o Currículo Lattes
%   Email    = {advisor1@domain},%% Opcional
%   Lattes   = {0000000000020001},%% Opcional
%   ORCID    = {0000-0000-0002-0001},%% Opcional (CHKTEX 8)
}
\Advisor{2}{
   Gender   = {Female},%% Ou {Female}
   Title    = {\ProfCall\ \PhDCall},%% {\<T>Call}; <T>: Prof/PhD/DSc/MSc/Eng
   Fullname = {Helyane Bronoski Borges},%% Conforme o Currículo Lattes
%   Email    = {advisor1@domain},%% Opcional
%   Lattes   = {0000000000020001},%% Opcional
%   ORCID    = {0000-0000-0002-0001},%% Opcional (CHKTEX 8)
}
%%%% Ficha catalográfica (somente para Teses e Dissertações em catálogo físico)
%%%% [Arg-1]: local (pasta) do PDF (./Pre-Textual/Extras/ por padrão).
%%%% {Arg-2}: nome do PDF (modelo em ./Pre-Textual/Extras/).
% \IndexCardPDF{doc-index-card.pdf}
\end{TitlePage}

%% Errata (elemento opcional; editar o {Arquivo} para alterar)
% %%%% ERRATA (ELEMENTO OPCIONAL)
%%
%% Lista dos erros ocorridos no texto, seguidos das devidas correções.

%% Errata
%% Ambiente {Errata*}: insere a autorreferência do documento.
\begin{Errata}%[Título Alternativo]%% Substitui o título padrão
%%%% Formato (com \midrule entre linhas): Página(s) & Onde se lê & Leia-se \\
\labelcpageref{err:chpt-1,err:chpt-2,err:chpt-3,err:chpt-4,err:chpt-5,err:chpt-6} &
capítulo{(s)}                                                                     &
seção{(ões)} primária{(s)}                                                        \\
\midrule%
\pageref{err:ssect}         &
subseção{(ões)}             &
seção{(ões)} terciária{(s)} \\
\end{Errata}


%% Folha de aprovação
%% Contém os elementos essenciais à aprovação do trabalho (sem as assinaturas).
%%%% Opção 1: gerada por meio do pacote utfpr-thesis.
%%%% Ambiente {ApprovalPage*}: insere a titulação após o nome do membro.
\begin{ApprovalPage}%% Argumentos (4):
% [brazilian]%% Idioma original ou primário (brazilian ou english)
{Month Day, Year}%{August 14, \YearNum}%% Data de aprovação (dia, mês por extenso e ano)
{14/08/\YearNum}%% Data de aprovação (forma abreviada; mestrado e doutorado)
{\linewidth}%% Largura de linha de assinatura (graduação e especialização)
%%%%%% Descrição do trabalho (padrão; alterar se necessário)
\DocumentTypeName\ presented as requirement to obtain the title of \StudentsTitlesList\ em \CourseName\ da \ifbool{MakeAcr}{\intldescr{UTFPR} (\intl{UTFPR})}{\UTFPRName\ (UTFPR)}.
%%%%%% Membro(s) da banca (de 3 a 6): {Número}; {Dados}
\Member{1}{%
  Gender      = {Female},%% Ou {Female}
  Title       = {\ProfCall\ \PhDCall},%% {\<T>Call}; <T>: Prof/PhD/DSc/MSc/Eng
  Fullname    = {Simone Nasser Matos (Advisor)},%% Conforme o Currículo Lattes
  Institution = {\UTFPRName},%% Nome completo e por extenso
}
\Member{2}{%
  Gender      = {Female},%% Ou {Male}
  Title       = {\ProfCall\ \PhDCall},%% {\<T>Call}; <T>: Prof/PhD/DSc/MSc/Eng
  Fullname    = {Leticia Mara Peres},%% Conforme o Currículo Lattes
  Institution = {Federal University of Paraná},%% Nome completo e por extenso
}
\Member{3}{%
  Gender      = {Female},%% Ou {Male}
  Title       = {\ProfCall\ \PhDCall},%% {\<T>Call}; <T>: Prof/PhD/DSc/MSc/Eng
  Fullname    = {Simone do Rocio Senger de Souza},%% Conforme o Currículo Lattes
  Institution = {University of São Paulo},%% Nome completo e por extenso
}
\end{ApprovalPage}
%%%% Opção 2: gerada a partir do Sistema Acadêmico ou da secretaria.
%%%% [Arg-1]: local (pasta) do PDF (./Pre-Textual/Extras/ por padrão).
%%%% {Arg-2}: nome do PDF (modelos em ./Pre-Textual/Extras/).
% \ApprovalPagePDF{doc-approval-page.pdf}

%% Dedicatória (elemento opcional; editar o {Arquivo} para alterar)
% %%%% DEDICATÓRIA (ELEMENTO OPCIONAL)
%%
%% Texto (pessoal) em que se presta homenagem ou se dedica o trabalho.

%% Dedicatória
\begin{Dedication}%% Argumentos (2):
% [0.5\textheight]%% Deslocamento vertical a partir da margem superior
% [Título]%% Não se aplica
%%%% Texto
Dedico este trabalho a minha família e aos meus amigos, pelos momentos de ausência.
\end{Dedication}


%% Agradecimentos (elemento opcional; editar o {Arquivo} para alterar)
%%%% AGRADECIMENTOS (ELEMENTO OPCIONAL)
%%
%% Texto (pessoal) em que se fazem agradecimentos dirigidos àqueles que
%% contribuíram de maneira relevante à elaboração do trabalho.

%% Agradecimentos
\begin{Acknowledgments}%[Título Alternativo]%% Substitui o título padrão
O presente trabalho não poderia ser finalizado sem a ajuda de diversas pessoas e/ou instituições às quais presto meus agradecimentos.
Certamente, esses parágrafos não abrangem todas as pessoas que fizeram parte dessa importante fase de minha vida.
Portanto, desde já peço desculpas àquelas que não estão presentes entre estas palavras, mas elas podem estar certas que fazem parte do meu pensamento com minha gratidão.\par%
A minha família, pelo carinho, incentivo e total apoio em todos os momentos da minha vida.\par%
Ao meu orientador, que me mostrou os caminhos a serem seguidos e pela confiança depositada.\par%
A todos os professores e colegas do curso, que ajudaram direta e indiretamente na realização e/ou conclusão deste trabalho.\par%
Aos demais que de alguma forma contribuíram para meu crescimento pessoal e profissional.\par%
%%%% À agência de fomento (último): {Nome}; {Número/Código do Fomento}
\FundingAgency{%
  da \ifbool{MakeAcr}{\intldescr{CAPES}}{Coordenação de Aperfeiçoamento de Pessoal de Nível Superior} \textemdash\ Brasil (\ifbool{MakeAcr}{\intl{CAPES}}{CAPES})%% CAPES
%   do \ifbool{MakeAcr}{\intl{CNPq}, \intldescr{CNPq}}{CNPq, Conselho Nacional de Desenvolvimento Científico e Tecnológico} \textemdash\ Brasil%% CNPq
%   da Fundação Araucária \textemdash\ Brasil%% FA
%   da \ifbool{MakeAcr}{\intl{UTFPR}, \intldescr{UTFPR}}{UTFPR, \UTFPRName} \textemdash\ Brasil%% UTFPR
}{%
  \textemdash\ Código de Financiamento 001%% CAPES
%   (\No\ de processo)%% CNPq
%   (\No\ de edital, financiamento, processo ou projeto)%% FA
%   (\No\ de edital, financiamento, processo ou projeto)%% UTFPR
}
\end{Acknowledgments}


%% Epígrafe (elemento opcional; editar o {Arquivo} para alterar)
% %%%% EPÍGRAFE (ELEMENTO OPCIONAL)
%%
%% Texto em que se apresenta uma citação, seguida de indicação de autoria,
%% relacionada com a matéria tratada no corpo do trabalho.

%% Opção 1: baseada na ABNT NBR 10520 (citações diretas curtas e longas).
%% Ambiente {Epigraph*}: remove o formato de citação direta longa.
\begin{Epigraph}%% Argumentos (2):
% [0.5\textheight]%% Deslocamento vertical a partir da margem superior
% [Título]%% Não se aplica
%%%% Epígrafes nos idiomas primário (texto) e original (nota de rodapé)
%%%% [Arg-1]: idioma (brazilian ou english).
%%%% {Arg-2}: autoria.
%%%% {Arg-3}: texto.
%%%% [Arg-4]: nota de rodapé.
\Citation[brazilian]{\cite[tradução própria]{Einstein1921}}{%
  Até onde as leis da matemática se referem à realidade, não são certas; e até onde são certas, não se referem à realidade
}[%
  \Citation[english]{\cite{Einstein1921}}{%
    As far as the laws of mathematics refer to reality, they are not certain; and as far as they are certain, they do not refer to reality
  }.
].
\par%
\Citation[brazilian]{\cite[\ppno~37, tradução própria]{Asimov1950}}{%
  Primeira Lei: um robô não pode ferir um ser humano ou, por omissão, permitir que um ser humano sofra algum mal.
  Segunda Lei: um robô deve obedecer às ordens que lhe sejam dadas por seres humanos, exceto nos casos em que tais ordens contrariem a Primeira Lei.
  Terceira Lei: um robô deve proteger sua própria existência desde que tal proteção não entre em conflito com a Primeira ou Segunda Leis
}[%
  \Citation[english]{\cite[37]{Asimov1950}}{%
    First Law: a robot may not injure a human being or, through inaction, allow a human being to come to harm.
    Second Law: a robot must obey the orders given it by human beings except where such orders would conflict with the First Law.
    Third Law: a robot must protect its own existence as long as such protection does not conflict with the First or Second Laws
  }.
].
\end{Epigraph}

%% Opção 2: baseada na classe de documento memoir.
% \begin{Epigraphs}%% Argumentos (2):
% [0.5\textheight]%% Deslocamento vertical a partir da margem superior
% [Título]%% Não se aplica
%%%% Epígrafes nos idiomas primário (texto) e original (nota de rodapé)
%%%% [Arg-1]: idioma (brazilian ou english).
%%%% {Arg-2}: autoria.
%%%% {Arg-3}: texto.
%%%% [Arg-4]: nota de rodapé.
% \EpiItem[brazilian]{\cite[tradução própria]{Einstein1921}}{%
%   Até onde as leis da matemática se referem à realidade, não são certas; e até onde são certas, não se referem à realidade.
% }[%
%   \Citation[english]{\cite{Einstein1921}}{%
%     As far as the laws of mathematics refer to reality, they are not certain; and as far as they are certain, they do not refer to reality
%   }.
% ]
% \end{Epigraphs}


%% Resumo
%% Apresentação concisa dos pontos relevantes de um texto, fornecendo uma visão
%% rápida e clara do conteúdo e das conclusões do trabalho.
%% Ambiente {Abstract*}: insere a autorreferência do documento.
%%%% Estilo de fonte da chamada das palavras-chave (opcional)
% \KeywordsCallFormat{\bfseries}%% Texto normal por padrão
%%%% Palavras-chave (de 3 a 6): {Número}; {Em Português}; {In English}
\Keyword{1}{refatoração de software}{software refactoring}
\Keyword{2}{padrões de projetos}{design patterns}
\Keyword{3}{microserviços}{microservices}
\Keyword{4}{processo semiautomatizado}{semi-automated process}
%%%% Em língua vernácula (idioma primário)
\begin{Abstract}[brazilian]%% Idioma (brazilian ou english)
Refatoração é um meio de melhorar o código-fonte sem alterar a sua funcionalidade, removendo code smells e tornando-o mais flexível e legível. Dentre as técnicas de refatoração, existe a refatoração por padrões de projetos que permite criar um código com maior qualidade em relação a atributos como reusabilidade, flexibilidade, entre outros. A ferramenta Refactoring and Measurement Tool (RMT) foi criada em sua primeira versão pelo Laboratório de Engenharia de Software e Inteligência Computacional (LESIC) e é capaz de ler um projeto escrito em linguagem java e detectar e aplicar padrões de projeto em linguagem Java. Para isso, ela contém a implementação de três métodos da literatura capazes de detectar e aplicar padrões de projeto. A dificuldade da primeira versão da RMT está relacionada em se incorporar em sua arquitetura uma quantidade maior de métodos. Este trabalho realizou a refatoração de código e de sua arquitetura. O processo de melhoramento da primeira versão da RMT abrangeu as fases de: análise, restruturação, testes, refatoração e avaliação. A arquitetura da RMT foi modificada usando microserviços assíncronos e nativos na nuvem para melhorar a performance, disponibilidade e escalabilidade, desacoplando as responsabilidades. Como resultado criou-se a versão da RMT 2.0 que contém um escalonamento horizontal para melhorar a performance sobre demanda, combinando com aplicações em nuvem. Os testes criados facilitam aos desenvolvedores realizarem modificações para estender a ferramenta. O código-fonte foi simplificado para trazer melhor performance a ferramenta e melhorar a experiência do desenvolvedor que deseja contribuir com a ferramenta. O processo de execução local da ferramenta foi alterado, trazendo melhora na facilidade do mecanismo de execução, podendo ser executada com containers. A análise dos resultados evidencia uma redução de 63.64\% de tempo de execução na refatoração dos projetos testados em relação a ferramenta original.

\end{Abstract}
%%%% Em língua estrangeira (idioma secundário; para divulgação internacional)
\begin{Abstract}[english]%% Idioma (brazilian ou english)
Refactoring enhances the integrity of the source code without altering its functionality, eliminating code smells while improving its flexibility and readability. Among the various refactoring techniques, using design patterns facilitates the development of higher-quality code with enhanced attributes such as reusability and flexibility. The Refactoring and Measurement Tool (RMT) was initially developed by the Software Engineering and Computational Intelligence Laboratory (LESIC) and is capable of parsing Java projects to detect and implement design patterns. This functionality is achieved by integrating three established methodologies from the literature. The primary challenge of the first version of RMT is its limited capacity to incorporate additional methodologies into its architectural framework. This work has done a comprehensive refactoring of both the codebase and the architecture. The enhancements to the initial version of RMT were executed in phases that included analysis, restructuring, testing, refactoring, and evaluation. The RMT architecture was reengineered using asynchronous, cloud-native microservices to boost performance, availability, and scalability while segregating responsibilities. Consequently, RMT 2.0 was developed, featuring horizontal scaling to meet the performance demands associated with cloud integration. The developed testing facilitates developer modifications to extend the tool's feature set.  The source code has been optimized to enhance the tool's performance, thereby improving the development experience for contributors. The tool's local execution process has been modified to streamline the execution mechanism by allowing container-based deployment. Empirical analysis of the results indicates a 63.64\% reduction in execution time during project refactoring compared to the original tool.
\end{Abstract}
