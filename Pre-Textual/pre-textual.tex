%%%% ELEMENTOS PRÉ-TEXTUAIS
%%
%% Parte que antecede o texto com informações que ajudam na identificação e na
%% utilização do trabalho.
%%
%% Observações:
%% 1. {Arg} argumento obrigatório de ambiente ou comando.
%% 2. [Arg] argumento opcional de ambiente ou comando.

%% Folha de rosto
%% Contém os elementos essenciais à identificação do trabalho, além de uma
%% licença Creative Commons (https://creativecommons.org/choose/).
%% Ambiente {TitlePage*}: aplica caixa alta no título em idioma secundário.
\begin{TitlePage}%% Argumentos (2):
[BY]%% Tipo de licença (BY, BY-SA, BY-ND, BY-NC, BY-NC-SA ou BY-NC-ND)
% [Texto da licença]%% Substitui o texto padrão para cada tipo de licença
%%%% Descrição do trabalho (padrão; alterar se necessário)
\DocumentTypeName\ presented as requirement to obtain the title of \StudentsTitlesList\ in \CourseName\ from \ifbool{MakeAcr}{\intldescr{UTFPR} (\intl{UTFPR})}{\UTFPRName\ (UTFPR)}.
%%%% Orientador(es) (de 1 a 3): {Número}; {Dados}
\Advisor{1}{%
   Gender   = {Female},%% Ou {Female}
   Title    = {\ProfCall\ \PhDCall},%% {\<T>Call}; <T>: Prof/PhD/DSc/MSc/Eng
   Fullname = {Simone Nasser Matos},%% Conforme o Currículo Lattes
%   Email    = {advisor1@domain},%% Opcional
%   Lattes   = {0000000000020001},%% Opcional
%   ORCID    = {0000-0000-0002-0001},%% Opcional (CHKTEX 8)
}
%%%% Ficha catalográfica (somente para Teses e Dissertações em catálogo físico)
%%%% [Arg-1]: local (pasta) do PDF (./Pre-Textual/Extras/ por padrão).
%%%% {Arg-2}: nome do PDF (modelo em ./Pre-Textual/Extras/).
% \IndexCardPDF{doc-index-card.pdf}
\end{TitlePage}

%% Errata (elemento opcional; editar o {Arquivo} para alterar)
% %%%% ERRATA (ELEMENTO OPCIONAL)
%%
%% Lista dos erros ocorridos no texto, seguidos das devidas correções.

%% Errata
%% Ambiente {Errata*}: insere a autorreferência do documento.
\begin{Errata}%[Título Alternativo]%% Substitui o título padrão
%%%% Formato (com \midrule entre linhas): Página(s) & Onde se lê & Leia-se \\
\labelcpageref{err:chpt-1,err:chpt-2,err:chpt-3,err:chpt-4,err:chpt-5,err:chpt-6} &
capítulo{(s)}                                                                     &
seção{(ões)} primária{(s)}                                                        \\
\midrule%
\pageref{err:ssect}         &
subseção{(ões)}             &
seção{(ões)} terciária{(s)} \\
\end{Errata}


%% Folha de aprovação
%% Contém os elementos essenciais à aprovação do trabalho (sem as assinaturas).
%%%% Opção 1: gerada por meio do pacote utfpr-thesis.
%%%% Ambiente {ApprovalPage*}: insere a titulação após o nome do membro.
\begin{ApprovalPage}%% Argumentos (4):
% [brazilian]%% Idioma original ou primário (brazilian ou english)
{DD de mmmmmm de \YearNum}%% Data de aprovação (dia, mês por extenso e ano)
{DD/MM/\YearNum}%% Data de aprovação (forma abreviada; mestrado e doutorado)
{\linewidth}%% Largura de linha de assinatura (graduação e especialização)
%%%%%% Descrição do trabalho (padrão; alterar se necessário)
\DocumentTypeName\ apresentad\ifbool{Graduate}{a}{o} como requisito para obtenção do título de \StudentsTitlesList\ em \CourseName\ da \ifbool{MakeAcr}{\intldescr{UTFPR} (\intl{UTFPR})}{\UTFPRName\ (UTFPR)}.
%%%%%% Membro(s) da banca (de 3 a 6): {Número}; {Dados}
% \Member{1}{%
%   Gender      = {Male},%% Ou {Female}
%   Title       = {\ProfCall\ \PhDCall},%% {\<T>Call}; <T>: Prof/PhD/DSc/MSc/Eng
%   Fullname    = {Prenome{(s)} Sobrenome-C1},%% Conforme o Currículo Lattes
%   Institution = {Instituição (Membro-C1)},%% Nome completo e por extenso
% }
% \Member{2}{%
%   Gender      = {Male},%% Ou {Female}
%   Title       = {\ProfCall\ \PhDCall},%% {\<T>Call}; <T>: Prof/PhD/DSc/MSc/Eng
%   Fullname    = {Prenome{(s)} Sobrenome-C2},%% Conforme o Currículo Lattes
%   Institution = {Instituição (Membro-C2)},%% Nome completo e por extenso
% }
% \Member{3}{%
%   Gender      = {Male},%% Ou {Female}
%   Title       = {\ProfCall\ \PhDCall},%% {\<T>Call}; <T>: Prof/PhD/DSc/MSc/Eng
%   Fullname    = {Prenome{(s)} Sobrenome-C3},%% Conforme o Currículo Lattes
%   Institution = {Instituição  (Membro-C3)},%% Nome completo e por extenso
% }
\end{ApprovalPage}
%%%% Opção 2: gerada a partir do Sistema Acadêmico ou da secretaria.
%%%% [Arg-1]: local (pasta) do PDF (./Pre-Textual/Extras/ por padrão).
%%%% {Arg-2}: nome do PDF (modelos em ./Pre-Textual/Extras/).
% \ApprovalPagePDF{doc-approval-page.pdf}

%% Dedicatória (elemento opcional; editar o {Arquivo} para alterar)
% %%%% DEDICATÓRIA (ELEMENTO OPCIONAL)
%%
%% Texto (pessoal) em que se presta homenagem ou se dedica o trabalho.

%% Dedicatória
\begin{Dedication}%% Argumentos (2):
% [0.5\textheight]%% Deslocamento vertical a partir da margem superior
% [Título]%% Não se aplica
%%%% Texto
Dedico este trabalho a minha família e aos meus amigos, pelos momentos de ausência.
\end{Dedication}


%% Agradecimentos (elemento opcional; editar o {Arquivo} para alterar)
% %%%% AGRADECIMENTOS (ELEMENTO OPCIONAL)
%%
%% Texto (pessoal) em que se fazem agradecimentos dirigidos àqueles que
%% contribuíram de maneira relevante à elaboração do trabalho.

%% Agradecimentos
\begin{Acknowledgments}%[Título Alternativo]%% Substitui o título padrão
O presente trabalho não poderia ser finalizado sem a ajuda de diversas pessoas e/ou instituições às quais presto meus agradecimentos.
Certamente, esses parágrafos não abrangem todas as pessoas que fizeram parte dessa importante fase de minha vida.
Portanto, desde já peço desculpas àquelas que não estão presentes entre estas palavras, mas elas podem estar certas que fazem parte do meu pensamento com minha gratidão.\par%
A minha família, pelo carinho, incentivo e total apoio em todos os momentos da minha vida.\par%
Ao meu orientador, que me mostrou os caminhos a serem seguidos e pela confiança depositada.\par%
A todos os professores e colegas do curso, que ajudaram direta e indiretamente na realização e/ou conclusão deste trabalho.\par%
Aos demais que de alguma forma contribuíram para meu crescimento pessoal e profissional.\par%
%%%% À agência de fomento (último): {Nome}; {Número/Código do Fomento}
\FundingAgency{%
  da \ifbool{MakeAcr}{\intldescr{CAPES}}{Coordenação de Aperfeiçoamento de Pessoal de Nível Superior} \textemdash\ Brasil (\ifbool{MakeAcr}{\intl{CAPES}}{CAPES})%% CAPES
%   do \ifbool{MakeAcr}{\intl{CNPq}, \intldescr{CNPq}}{CNPq, Conselho Nacional de Desenvolvimento Científico e Tecnológico} \textemdash\ Brasil%% CNPq
%   da Fundação Araucária \textemdash\ Brasil%% FA
%   da \ifbool{MakeAcr}{\intl{UTFPR}, \intldescr{UTFPR}}{UTFPR, \UTFPRName} \textemdash\ Brasil%% UTFPR
}{%
  \textemdash\ Código de Financiamento 001%% CAPES
%   (\No\ de processo)%% CNPq
%   (\No\ de edital, financiamento, processo ou projeto)%% FA
%   (\No\ de edital, financiamento, processo ou projeto)%% UTFPR
}
\end{Acknowledgments}


%% Epígrafe (elemento opcional; editar o {Arquivo} para alterar)
% %%%% EPÍGRAFE (ELEMENTO OPCIONAL)
%%
%% Texto em que se apresenta uma citação, seguida de indicação de autoria,
%% relacionada com a matéria tratada no corpo do trabalho.

%% Opção 1: baseada na ABNT NBR 10520 (citações diretas curtas e longas).
%% Ambiente {Epigraph*}: remove o formato de citação direta longa.
\begin{Epigraph}%% Argumentos (2):
% [0.5\textheight]%% Deslocamento vertical a partir da margem superior
% [Título]%% Não se aplica
%%%% Epígrafes nos idiomas primário (texto) e original (nota de rodapé)
%%%% [Arg-1]: idioma (brazilian ou english).
%%%% {Arg-2}: autoria.
%%%% {Arg-3}: texto.
%%%% [Arg-4]: nota de rodapé.
\Citation[brazilian]{\cite[tradução própria]{Einstein1921}}{%
  Até onde as leis da matemática se referem à realidade, não são certas; e até onde são certas, não se referem à realidade
}[%
  \Citation[english]{\cite{Einstein1921}}{%
    As far as the laws of mathematics refer to reality, they are not certain; and as far as they are certain, they do not refer to reality
  }.
].
\par%
\Citation[brazilian]{\cite[\ppno~37, tradução própria]{Asimov1950}}{%
  Primeira Lei: um robô não pode ferir um ser humano ou, por omissão, permitir que um ser humano sofra algum mal.
  Segunda Lei: um robô deve obedecer às ordens que lhe sejam dadas por seres humanos, exceto nos casos em que tais ordens contrariem a Primeira Lei.
  Terceira Lei: um robô deve proteger sua própria existência desde que tal proteção não entre em conflito com a Primeira ou Segunda Leis
}[%
  \Citation[english]{\cite[37]{Asimov1950}}{%
    First Law: a robot may not injure a human being or, through inaction, allow a human being to come to harm.
    Second Law: a robot must obey the orders given it by human beings except where such orders would conflict with the First Law.
    Third Law: a robot must protect its own existence as long as such protection does not conflict with the First or Second Laws
  }.
].
\end{Epigraph}

%% Opção 2: baseada na classe de documento memoir.
% \begin{Epigraphs}%% Argumentos (2):
% [0.5\textheight]%% Deslocamento vertical a partir da margem superior
% [Título]%% Não se aplica
%%%% Epígrafes nos idiomas primário (texto) e original (nota de rodapé)
%%%% [Arg-1]: idioma (brazilian ou english).
%%%% {Arg-2}: autoria.
%%%% {Arg-3}: texto.
%%%% [Arg-4]: nota de rodapé.
% \EpiItem[brazilian]{\cite[tradução própria]{Einstein1921}}{%
%   Até onde as leis da matemática se referem à realidade, não são certas; e até onde são certas, não se referem à realidade.
% }[%
%   \Citation[english]{\cite{Einstein1921}}{%
%     As far as the laws of mathematics refer to reality, they are not certain; and as far as they are certain, they do not refer to reality
%   }.
% ]
% \end{Epigraphs}


%% Resumo
%% Apresentação concisa dos pontos relevantes de um texto, fornecendo uma visão
%% rápida e clara do conteúdo e das conclusões do trabalho.
%% Ambiente {Abstract*}: insere a autorreferência do documento.
%%%% Estilo de fonte da chamada das palavras-chave (opcional)
% \KeywordsCallFormat{\bfseries}%% Texto normal por padrão
%%%% Palavras-chave (de 3 a 6): {Número}; {Em Português}; {In English}
\Keyword{1}{refatoração}{refactoring}
\Keyword{2}{refatoração de software}{software refactoring}
\Keyword{3}{padrões de projetos}{design patterns}
\Keyword{4}{nuvem}{cloud}
\Keyword{5}{microserviços}{microservices}
%%%% Em língua vernácula (idioma primário)
\begin{Abstract}[brazilian]%% Idioma (brazilian ou english)
Refatoração é um meio de melhorar o código-fonte sem alterar a sua funcionalidade, removendo code smells e o tornando mais flexível e legível. Dentre as técnicas de refatoração, existe a refatoração por padrões de projetos que permite criar um código com maior qualidade em relação a atributos como reusabilidade, flexibilidade, entre outros. A ferramenta Refactoring and Measurement Tool (RMT) foi criada em sua primeira versão pelo Laboratório de Engenharia de Software e Inteligência Computacional (LESIC) e é capaz de ler um projeto escrito em linguagem java e detectar e aplicar padrões de projeto. Para isto, ela contém a implementação de três métodos da literatura capazes de detectar e aplicar padrões de projeto. A dificuldade da primeira versão da RMT está relacionada em se incorporar em sua arquitetura uma quantidade maior de métodos. Este trabalho propõe a refatoração de sua arquitetura usando microserviços assíncronos e nativos na nuvem para melhorar a performance, disponibilidade e escalabilidade, desacoplamento as responsabilidades. Esta arquitetura foca em um escalonamento horizontal para melhorar a performance sobre demanda, combinando com aplicações em nuvem.
\end{Abstract}
%%%% Em língua estrangeira (idioma secundário; para divulgação internacional)
\begin{Abstract}[english]%% Idioma (brazilian ou english)
Refactoring improves source code without altering its functionality, removes code smells, and makes it more flexible and readable. Pattern-based refactoring is one of the refactoring techniques, which allows the creation of code with higher quality concerning attributes such as reusability and flexibility, among others. The Refactoring and Measurement Tool (RMT) was created in its first version by the Software Engineering and Computational Intelligence Laboratory (LESIC), which can read a project written in Java and detect and apply design patterns. To do so, it contains the implementation of three methods from the literature capable of detecting and applying design patterns. The difficulty of the first version of RMT is related to incorporating more methods into its architecture. This work proposes the refactoring of its architecture using asynchronous and cloud-native microservices to improve performance, availability, scalability, and decoupling responsibilities. This architecture focuses on horizontal scaling to improve on-demand performance combined with cloud applications.
\end{Abstract}
